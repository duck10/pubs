\documentclass[final]{beamer}
\usepackage[paperwidth=45in, paperheight=40in, margin=1in]{geometry}
\usepackage{graphicx} % For images
\usepackage{multicol} % Multi-column formatting
\usepackage{color} % Text coloring
\usepackage{amsmath, amssymb} % Math symbols
\usepackage{tcolorbox} % For boxed text elements

\usetheme{Berlin} % Beamer theme
\usecolortheme{dolphin} % Color theme

\begin{document}
	
	\begin{frame}[plain]
		\title{\textbf{Unit Cells in Space or Spaces for Unit Cells}}
		\author{Your Name}
		\date{Your Institution}
		\maketitle
	\end{frame}
	
	\begin{frame}{Introduction}
		\begin{multicols}{2}
			Unit cells are the fundamental repeating units in crystalline structures. Their spatial arrangement plays a crucial role in determining material properties.
		\end{multicols}
	\end{frame}
	
	\begin{frame}{Methods}
		\begin{multicols}{2}
			We examine crystallographic reductions, including transformations such as **Niggli reduction** using 3×3 matrices.
		\end{multicols}
	\end{frame}
	
	\begin{frame}{Results}
		\centering
		\includegraphics[width=0.9\linewidth]{example-figure.png} % Replace with your actual file
	\end{frame}
	
	\begin{frame}{Conclusion}
		\begin{multicols}{2}
			The unit cell transformation methods explored here provide insights into space group classifications and lattice symmetry.
		\end{multicols}
	\end{frame}
	
\end{document}
