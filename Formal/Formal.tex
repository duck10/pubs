%------------------------------------------------------------------------------
% Template file for the submission of papers to IUCr journals in LaTeX2e
% using the iucr document class
% Copyright 1999-2013 International Union of Crystallography
% Version 1.6 (28 March 2013)
%------------------------------------------------------------------------------

%xxxxxx just for testing github % 2
%xxxxxx just for testing github % 3

\documentclass[preprint]{iucr}              % DO NOT DELETE THIS LINE
\usepackage{amssymb}
\usepackage[fleqn]{amsmath}
%\usepackage{bm}
\usepackage{graphicx}
\usepackage{tabularx}
\usepackage{booktabs}
%\usepackage{calligra}
\usepackage{array}
\DeclareMathAlphabet{\mathcalligra}{T1}{calligra}{m}{n}
\def\mathbi#1{\textbf{\em #1}}
\numberwithin{equation}{section}
%\DeclareMathSymbol{\Gamma}{\mathalpha}{letters}{"00}
%\DeclareMathSymbol{\Lambda}{\mathalpha}{letters}{"03}
%\DeclareMathSymbol{\Omega}{\mathalpha}{letters}{"0A}
%\DeclareMathAlphabet{\mathitbf}{OML}{cmm}{b}{it}
\hyphenation{Niggli}
\def\mathbi#1{\textbf{\em #1}}
%\numberwithin{equation}{section}
%\DeclareMathSymbol{\Gamma}{\mathalpha}{letters}{"00}
%\DeclareMathSymbol{\Lambda}{\mathalpha}{letters}{"03}
%\DeclareMathSymbol{\Omega}{\mathalpha}{letters}{"0A}
%\DeclareMathAlphabet{\mathitbf}{OML}{cmm}{b}{it}
\usepackage{color}
\usepackage{ulem}
\usepackage{url}
\usepackage{yfonts}
%\usepackage{xr-hyper}
%\usepackage[draft]{hyperref}
%\usepackage{bibentry}
\newcommand{\SVI}[0]{$\bf{S^{6}}$}
\newcommand{\GVI}[0]{$\bf{G^{6}}$}
\newcommand{\CIII}[0]{$\bf{C^{3}}$}
\newcommand{\DVII}[0]{$\bf{D^{7}}$}
\newcommand{\VVII}[0]{$\bf{V^{7}}$}

\newcommand{\vdotv}[2]{${{\bf #1 \cdot #2}}$}
\newcommand{\Imaginary}[0]{\mathcal{I}}
\newcommand{\Real}[0]{\mathcal{R}}
\newcommand{\Exchange}[0]{$\mathcal{X}$}

\newcommand{\nounderline}[3]{\!\!\!\!\!\!\!\!\!#1,&\!\!\!\!\!\!\!\!\!#2,&\!\!\!\!\!\!\!\!\!#3}
\newcommand{\underlineab}[3]{\!\!\!\!\!\!\!\!\!\!\!\!\!\!\!\!\!\!\!\!\!\!\!\!\Exchange{}(#1),&\!\!\!\!\!\!\!\!\!\!\!\!\!\!\!\!\!\!\!\!\!\!\!\!\Exchange{}(#2),&\!\!\!\!\!\!\!\!\!#3}
\newcommand{\underlineac}[3]{\!\!\!\!\!\!\!\!\!\!\!\!\!\!\!\!\!\!\!\!\!\!\!\!\Exchange{}(#1),&\!\!\!\!\!\!\!\!\!\!\!\!\!\!\!\!\!\!\!\!\!\!\!\!#2,&\!\!\!\!\!\!\!\!\!\Exchange{}(#3)}
\newcommand{\underlinebc}[3]{\!\!\!\!\!\!\!\!\!\!\!\!\!\!\!\!\!\!\!\!\!\!\!\!#1,&\!\!\!\!\!\!\!\!\!\Exchange{}(#2),&\!\!\!\!\!\!\!\!\!\!\!\!\!\!\!\!\!\!\!\!\!\!\!\!\Exchange{}(#3)}

\newcommand{\scalar}[1]{$#1$}

\newcommand{\scalarsub}[2]{$#1_#2$}

%-------------------------------------------------------------------------
% Information about journal to which submitted
%-------------------------------------------------------------------------
\journalcode{A}              % Indicate the journal to which submitted
%   A - Acta Crystallographica Section A
%   B - Acta Crystallographica Section B
%   C - Acta Crystallographica Section C
%   D - Acta Crystallographica Section D
%   E - Acta Crystallographica Section E
%   F - Acta Crystallographica Section F
%   J - Journal of Applied Crystallography
%   M - IUCrJ
%   S - Journal of Synchrotron Radiation
\makeatletter
\font\dummyft@=dummy \relax
\makeatother


\begin{document}                  % DO NOT DELETE THIS LINE
	
	%-------------------------------------------------------------------------
	% The introductory (header) part of the paper
	%-------------------------------------------------------------------------
	
	% The title of the paper. Use \shorttitle to indicate an abbreviated title
	% for use in running heads (you will need to uncomment it).
	
	% Authors' names and addresses. Use \cauthor for the main (contact) author.
	% Use \author for all other authors. Use \aff for authors' affiliations.
	% Use lower-case letters in square brackets to link authors to their
	% affiliations; if there is only one affiliation address, remove the [a].
	
	% Use \vita if required to give biographical details (for authors of
	% invited review papers only). Uncomment it.
	
	% lca IUCr id IUCr6401
	%\vita{Author's biography}
	
	% Keywords (required for Journal of Synchrotron Radiation only)
	% Use the \keyword macro for each word or phrase, e.g. 
	% \keyword{X-ray diffraction}\keyword{muscle}
	
	
	% PDB and NDB reference codes for structures referenced in the article and
	% deposited with the Protein Data Bank and Nucleic Acids Database (Acta
	% Crystallographica Section D). Repeat for each separate structure e.g
	% \PDBref[dethiobiotin synthetase]{1byi} \NDBref[d(G$_4$CGC$_4$)]{ad0002}
	
	%\PDBref[optional name]{refcode}
	%\NDBref[optional name]{refcode}
	
	%-------------------------------------------------------------------------
	% The introductory (header) part of the paper
	%-------------------------------------------------------------------------
	
	% The title of the paper. Use \shorttitle to indicate an abbreviated title
	% for use in running heads (you will need to uncomment it).
	{\LARGE \emph{\today}} \\
	\title{Proving a measure of lattice differences}
	%\title{Note on the transformation of three-space basis vectors to  corresponding matrix for Delaunay scalars}
	\shorttitle{What are unit cells}
	
	% Authors' names and addresses. Use \cauthor for the main (contact) author.
	% Use \author for all other authors. Use \aff for authors' affiliations.
	% Use lower-case letters in square brackets to link authors to their
	% affiliations; if there is only one affiliation address, remove the [a].
	
	
	\cauthor[a]{Lawrence C.}{Andrews}{larry6640995@gmail.com}{}
	\author[b]{Herbert J.}{Bernstein}
	
	\aff[a]{Ronin Institute, 9515 NE 137th St, Kirkland, WA, 98034-1820 \country{USA}}
	\aff[b]{Ronin Institute, c/o NSLS-II, Brookhaven National Laboratory, Upton, NY, 11973 \country{USA}}
	
	% Use \shortauthor to indicate an abbreviated author list for use in
	% running heads (you will need to uncomment it).
	
	\shortauthor{Andrews and Bernstein}
	
	% Use \vita if required to give biographical details (for authors of
	% invited review papers only). Uncomment it.
	
	% lca IUCr id IUCr6401
	%\vita{Author's biography}
	
	% Keywords (required for Journal of Synchrotron Radiation only)
	% Use the \keyword macro for each word or phrase, e.g. 
	% \keyword{X-ray diffraction}\keyword{muscle}
	
	\keyword{lattice}
	\keyword{unit cells}
	\keyword{Delone}
	\keyword{Selling}
	
	% PDB and NDB reference codes for structures referenced in the article and
	% deposited with the Protein Data Bank and Nucleic Acids Database (Acta
	% Crystallographica Section D). Repeat for each separate structure e.g
	% \PDBref[dethiobiotin synthetase]{1byi} \NDBref[d(G$_4$CGC$_4$)]{ad0002}
	
	%\PDBref[optional name]{refcode}
	%\NDBref[optional name]{refcode}
	
	\maketitle                        % DO NOT DELETE THIS LINE
	
	\begin{synopsis}
		Unit cells
	\end{synopsis}
	\newcommand{\si}[0]{$s_1$}
	\newcommand{\sii}[0]{$s_2$}
	\newcommand{\siii}[0]{$s_3$}
	\newcommand{\siv}[0]{$s_4$}
	\newcommand{\sv}[0]{$s_5$}
	\newcommand{\svi}[0]{$s_6$}
	\newcommand{\Svec} [0] {\{\si, \sii, \siii, \siv, \sv, \svi \}}
	\newcommand{\SvecA} [0] {\{-\si, -\si+\sii, \si+\siii, \si+\sv, \si+\siv, \si+\svi \}}
	
	\newcommand{\OPES}[0]{$E^3toS^6$}
	\newcommand{\OPESS}[0]{$$E^3toS^6$$}
	\newcommand{\MSVI}[0]{$M_{S^{6}}$}
	\newcommand{\MEIII}[0]{$M_{E^{3}}$}
	\newcommand{\Plus}[0]{\mathcal{P}}	
	\newcommand{\Minus}[0]{\mathcal{M}}
	
	\newcommand{\ci}[0]{$c_1$}
	\newcommand{\cii}[0]{$c_2$}
	\newcommand{\ciii}[0]{$c_3$}
	
	
	\begin{abstract}
		Abstract Unit cells
		
		{\bf Note:}  In his later publications, Boris Delaunay used the Russian version of his surname, Delone.\\
		
		
	\end{abstract}
	% Appendices appear after the main body of the text. They are prefixed by
	% a single \appendix declaration, and are then structured just like the
	% body text.
	
	
	\section{Introduction}
	
	A recent article \cite{bright2023continuous}
	questioned the formal basis of the measures 
	of differences between lattices that are described by
	\citeasnoun{andrews2023measuring} (and references therein).
	
	Here we present a formal description of the above mentioned methods.
	
	\section{Finding a minimum distance}
	\label{finding}
	Consider some metric space \textbf{X} in which a distance can 
	be determined between any two objects in \textbf{X}.
	
	A distance in \textbf{X} is defined as 
	
	$d$ = $\big|$ $\bf{x}$-$\bf{y}$ $\big|$ for $\bf{x}$,$\bf{y}$ $\in$ $\bf{X}$.
	
	The question that immediately arises how to represent
	the object in \textbf{X}. The conventional choices are
	to choose either the basis vectors of the lattice,
	a description of the unit cell itself, or as objects
	in some space, such as ...
	
	The points in lattices in 2 dimensions are addressed
	by the operations of the ``modular group'', also
	named PSL(2,$\bf{Z}$). The corresponding 3 dimensional
	group is PSL(3,$\bf{Z}$).
	
	Invariants are known for lattices in two dimensions 
	(see \citeasnoun{jones1987complex}, page 176 and 
	following). \citeasnoun{bright2023continuous} have
	considered alternatives. However, invariants are
	not known for PSL(3,$\bf{Z}$) or in higher dimensions.
	
	To compute the distance between two lattices (represented
	in a metric space, with operators $g_{i}$ in the group $\bf{G}$), we define
	
	$d_{min}$ = min \bigg| $g_{i}\bf{y}-\bf{x}$ \bigg| for 
	$\bf{x}$,$\bf{y}$ $\in$ $\bf{X}$ and
	$\bf{g_i}$ $\in$ $\bf{G}$.
	
	\noindent
	We conjecture that 
		
	$dy_{min}$ = min \bigg| $g_{i}\bf{x}-\bf{y}$ \bigg| for 
	$\bf{x}$,$\bf{y}$ $\in$ $\bf{X}$ and
	$\bf{g_i}$ $\in$ $\bf{G}$.
	
	\noindent
	has the same value as $d_{min}$ above.
	
	\section{Computing the minimum distance}
	
	While Section \ref{finding} contains a definition 
	for a minimum distance, a problem still exists. The
	group $PSL(3,\bf{Z}$) is an infinite group. For 
	practicality, a finite algorithm is required.
	
	The first step is compute the distance between the
	two points of interest with $g_i$ as the identity 
	operation. From the group $\bf{G}$, a set of operators
	must be chosen. It should include those operations that
	do not change the norm of the affected point plus
	a sufficient set of the other operations. The generators
	of the $\bf{G}$ is a minimal set, but for efficiency
	an expanded set is useful. 
	
	The above set of operations must then be applied
	iteratively until the distance between the two
	points exceeds some limit.
	



	
	\section{Summary}
	
	Summary blah blah blah
	
	
	
	
	
	
%	\section{Availability of code}
%	
%	The $C^{++}$ ~code for \CIII{} and related 
%	software tools is available in github.com, in
%	\url{https://github.com/duck10/LatticeRepLib.git}.
%	The program CmdToC3 uses the required files.
	
	%\appendix
	
	
	%\section{blah blah blah -- Supplementary Material}
	\ack{{\bf Acknowledgements}}
	
	Careful copy-editing and corrections by Frances C. Bernstein are 
	gratefully acknowledged.
	Our thanks to Jean Jakoncic and Alexei Soares for 
	helpful conversations and access to data and facilities at 
	Brookhaven National Laboratory.
	
	\ack{{\bf Funding information}}      
	
	Funding for this research was provided in part by:  
	US Department of Energy Offices of Biological and 
	Environmental Research and of Basic Energy Sciences 
	(grant No. DE-AC02-98CH10886; grant No. E-SC0012704); 
	U.S. National Institutes of Health (grant No. P41RR012408; 
	grant No. P41GM103473; grant No. P41GM111244; 
	grant No. R01GM117126,
	grant No. 1R21GM129570); Dectris, Ltd.
	
	
	\bibliography{Reduced}
	
	\bibliographystyle{iucr}
	
	
	
	%-------------------------------------------------------------------------
	% TABLES AND FIGURES SHOULD BE INSERTED AFTER THE MAIN BODY OF THE TEXT
	%-------------------------------------------------------------------------
	
	% Simple tables should use the tabular environment according to this
	% model
	
	% Postscript figures can be included with multiple figure blocks
	
	%C:\Users\lca\Source\Repos\LatticeRepLib\x64\Debug>plotc3
	%; Graphical output SVG file =PLT__2023-03-07.13_43_35.svg
	%
	%C:\Users\lca\Source\Repos\LatticeRepLib\x64\Debug>cmdniggli | plotc3
	%; Graphical output SVG file =PLT__2023-03-07.13_44_06.svg
	%
	%C:\Users\lca\Source\Repos\LatticeRepLib\x64\Debug>cmddelone | plotc3
	%; Graphical output SVG file =PLT__2023-03-08.07_11_03.svg
	%
	%C:\Users\lca\Source\Repos\LatticeRepLib\x64\Debug>cmdniggli | cmdperturb 5 20 | plotc3
	%; Graphical output SVG file =PLT__2023-03-08.09_00_13.svg
	%
	%C:\Users\lca\Source\Repos\LatticeRepLib\x64\Debug>cmdniggli | cmdperturb 5 100 | plotc3
	%; Graphical output SVG file =PLT__2023-03-08.09_00_21.svg
	
\end{document}                    % DO NOT DELETE THIS LINE
%%%%%%%%%%%%%%%%%%%%%%%%%%%%%%%%%%%%%%%%%%%%%%%%%%%%%%%%%%%%%%%%%%%%%%%%%%%%%%
