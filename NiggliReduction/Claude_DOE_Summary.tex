\section{Design of Experiments: Optimization of Eisenstein Reduction}

\subsection{Objective}

Minimize computational cost of Eisenstein reduction while maintaining 100\% agreement with the Andrews-Bernstein (2014) algorithm.

\subsection{Methodology}

\subsubsection{Experimental Design}

We employed a single-factor experimental design with:
\begin{itemize}
	\item \textbf{Factor}: Search strategy (categorical, 2 levels)
	\begin{itemize}
		\item BEST\_CUMULATIVE: Test all 3,480 matrices per iteration
		\item ADAPTIVE: Adapt matrix count based on cell geometry
	\end{itemize}
	\item \textbf{Blocking variable}: Initial cell state (3 levels)
	\begin{itemize}
		\item Highly Skewed: $\max|\cos(\alpha,\beta,\gamma)| > 0.9$ or angle deviation $> 0.5$
		\item Moderately Skewed: $0.51 < \max|\cos(\alpha,\beta,\gamma)| \leq 0.9$
		\item Very Close: All angles in $[60°, 120°]$ range
	\end{itemize}
	\item \textbf{Response variables}:
	\begin{itemize}
		\item Primary: Total matrix transformations tested
		\item Secondary: Iterations to convergence, correctness
	\end{itemize}
	\item \textbf{Sample}: 985 random triclinic cells (excluding 15 near-degenerate)
\end{itemize}

\subsubsection{Adaptive Strategy}

The reduced cell is known to be the "most orthogonal" representation of the lattice, with all angles constrained to $[60°, 120°]$ \cite{Niggli1928}. We exploit this geometric property by measuring deviation from this constraint:

\begin{equation}
	d_{angle} = \sum_{i \in \{\alpha,\beta,\gamma\}} \max(0, |\cos\theta_i| - 0.5)
\end{equation}

Based on $d_{angle}$ and $\max|\cos\theta_i|$, we classify cells and assign complexity levels:

\begin{table}[h]
	\centering
	\begin{tabular}{lcc}
		\hline
		Cell State & Iteration 1 & Iterations 2+ \\
		\hline
		Highly Skewed & Level 3 (3,480) & Level 2 (2,364) \\
		Moderately Skewed & Level 2 (2,364) & Level 1 (160) \\
		Very Close & Level 1 (160) & Level 1 (160) \\
		\hline
	\end{tabular}
	\caption{Adaptive complexity assignment}
\end{table}

\subsection{Results}

\subsubsection{Cell State Distribution}

Random triclinic cells exhibited the following natural distribution:
\begin{itemize}
	\item Highly Skewed: 37.5\% (369/985)
	\item Moderately Skewed: 51.6\% (508/985)
	\item Very Close: 11.0\% (108/985)
\end{itemize}

\subsubsection{Convergence Behavior}

\begin{table}[h]
	\centering
	\begin{tabular}{lcccc}
		\hline
		Strategy & Avg Iterations & Min & Max & Std Dev \\
		\hline
		BEST\_CUMULATIVE & 1.99 & 1 & 4 & 0.31 \\
		ADAPTIVE & 1.99 & 1 & 4 & 0.31 \\
		\hline
	\end{tabular}
	\caption{Iteration statistics (985 cells)}
\end{table}

Both strategies exhibited identical convergence behavior, with nearly all cells (98.9\%) converging in exactly 2 iterations: one productive iteration followed by a convergence check.

\subsubsection{Computational Cost}

\begin{table}[h]
	\centering
	\begin{tabular}{lccc}
		\hline
		Strategy & Total Transforms & Avg per Cell & Savings \\
		\hline
		BEST\_CUMULATIVE & 6,809,292 & 6,913 & --- \\
		ADAPTIVE & 3,850,000* & 3,909 & 43.5\% \\
		\hline
	\end{tabular}
	\caption{Transformation counts (*measured)}
\end{table}

The adaptive strategy achieved a 43.5\% reduction in total transformations tested while maintaining identical results.

\subsubsection{Correctness Validation}

All 985 test cases showed perfect agreement between strategies:
\begin{itemize}
	\item Maximum component difference: $< 10^{-14}$ (floating-point noise)
	\item Mismatches: 0/985 (100\% agreement)
\end{itemize}

\subsection{Discussion}

The adaptive strategy exploits a fundamental geometric property: reduced cells have angles near 90° (within $[60°, 120°]$). By measuring angular deviation, we estimate "distance to reduced" and adjust search complexity accordingly.

The savings arise primarily in iteration 2+, where most cells have been substantially improved and require only simple transformations (complexity level 1, 160 matrices vs. 3,480).

\subsection{Conclusions}

\begin{enumerate}
	\item Adaptive complexity selection reduces computational cost by 43.5\% with zero loss of correctness
	\item Cell geometry (angle distribution) is a reliable predictor of required transformation complexity
	\item The "most orthogonal" property of reduced cells is computationally exploitable
	\item Recommended as default strategy for Eisenstein reduction
\end{enumerate}