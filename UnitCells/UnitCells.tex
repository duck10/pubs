%------------------------------------------------------------------------------
% Template file for the submission of papers to IUCr journals in LaTeX2e
% using the iucr document class
% Copyright 1999-2013 International Union of Crystallography
% Version 1.6 (28 March 2013)
%------------------------------------------------------------------------------

%xxxxxx just for testing github % 2
%xxxxxx just for testing github % 3

\documentclass[preprint]{iucr}              % DO NOT DELETE THIS LINE
\usepackage{amssymb}
\usepackage[fleqn]{amsmath}
%\usepackage{bm}
\usepackage{graphicx}
\usepackage{tabularx}
\usepackage{booktabs}
%\usepackage{calligra}
\usepackage{array}
\DeclareMathAlphabet{\mathcalligra}{T1}{calligra}{m}{n}
\def\mathbi#1{\textbf{\em #1}}
\numberwithin{equation}{section}
%\DeclareMathSymbol{\Gamma}{\mathalpha}{letters}{"00}
%\DeclareMathSymbol{\Lambda}{\mathalpha}{letters}{"03}
%\DeclareMathSymbol{\Omega}{\mathalpha}{letters}{"0A}
%\DeclareMathAlphabet{\mathitbf}{OML}{cmm}{b}{it}
\hyphenation{Niggli}
\def\mathbi#1{\textbf{\em #1}}
%\numberwithin{equation}{section}
%\DeclareMathSymbol{\Gamma}{\mathalpha}{letters}{"00}
%\DeclareMathSymbol{\Lambda}{\mathalpha}{letters}{"03}
%\DeclareMathSymbol{\Omega}{\mathalpha}{letters}{"0A}
%\DeclareMathAlphabet{\mathitbf}{OML}{cmm}{b}{it}
\usepackage{color}
\usepackage{ulem}
\usepackage{url}
\usepackage{yfonts}
%\usepackage{xr-hyper}
%\usepackage[draft]{hyperref}
%\usepackage{bibentry}
\newcommand{\SVI}[0]{$\bf{S^{6}}$}
\newcommand{\GVI}[0]{$\bf{G^{6}}$}
\newcommand{\CIII}[0]{$\bf{C^{3}}$}
\newcommand{\DVII}[0]{$\bf{D^{7}}$}
\newcommand{\VVII}[0]{$\bf{V^{7}}$}

\newcommand{\vdotv}[2]{${{\bf #1 \cdot #2}}$}
\newcommand{\Imaginary}[0]{\mathcal{I}}
\newcommand{\Real}[0]{\mathcal{R}}
\newcommand{\Exchange}[0]{$\mathcal{X}$}

\newcommand{\nounderline}[3]{\!\!\!\!\!\!\!\!\!#1,&\!\!\!\!\!\!\!\!\!#2,&\!\!\!\!\!\!\!\!\!#3}
\newcommand{\underlineab}[3]{\!\!\!\!\!\!\!\!\!\!\!\!\!\!\!\!\!\!\!\!\!\!\!\!\Exchange{}(#1),&\!\!\!\!\!\!\!\!\!\!\!\!\!\!\!\!\!\!\!\!\!\!\!\!\Exchange{}(#2),&\!\!\!\!\!\!\!\!\!#3}
\newcommand{\underlineac}[3]{\!\!\!\!\!\!\!\!\!\!\!\!\!\!\!\!\!\!\!\!\!\!\!\!\Exchange{}(#1),&\!\!\!\!\!\!\!\!\!\!\!\!\!\!\!\!\!\!\!\!\!\!\!\!#2,&\!\!\!\!\!\!\!\!\!\Exchange{}(#3)}
\newcommand{\underlinebc}[3]{\!\!\!\!\!\!\!\!\!\!\!\!\!\!\!\!\!\!\!\!\!\!\!\!#1,&\!\!\!\!\!\!\!\!\!\Exchange{}(#2),&\!\!\!\!\!\!\!\!\!\!\!\!\!\!\!\!\!\!\!\!\!\!\!\!\Exchange{}(#3)}

\newcommand{\scalar}[1]{$#1$}

\newcommand{\scalarsub}[2]{$#1_#2$}

%-------------------------------------------------------------------------
% Information about journal to which submitted
%-------------------------------------------------------------------------
\journalcode{A}              % Indicate the journal to which submitted
%   A - Acta Crystallographica Section A
%   B - Acta Crystallographica Section B
%   C - Acta Crystallographica Section C
%   D - Acta Crystallographica Section D
%   E - Acta Crystallographica Section E
%   F - Acta Crystallographica Section F
%   J - Journal of Applied Crystallography
%   M - IUCrJ
%   S - Journal of Synchrotron Radiation
\makeatletter
\font\dummyft@=dummy \relax
\makeatother


\begin{document}                  % DO NOT DELETE THIS LINE
	
	%-------------------------------------------------------------------------
	% The introductory (header) part of the paper
	%-------------------------------------------------------------------------
	
	% The title of the paper. Use \shorttitle to indicate an abbreviated title
	% for use in running heads (you will need to uncomment it).
	
	% Authors' names and addresses. Use \cauthor for the main (contact) author.
	% Use \author for all other authors. Use \aff for authors' affiliations.
	% Use lower-case letters in square brackets to link authors to their
	% affiliations; if there is only one affiliation address, remove the [a].
	
	% Use \vita if required to give biographical details (for authors of
	% invited review papers only). Uncomment it.
	
	% lca IUCr id IUCr6401
	%\vita{Author's biography}
	
	% Keywords (required for Journal of Synchrotron Radiation only)
	% Use the \keyword macro for each word or phrase, e.g. 
	% \keyword{X-ray diffraction}\keyword{muscle}
	
	
	% PDB and NDB reference codes for structures referenced in the article and
	% deposited with the Protein Data Bank and Nucleic Acids Database (Acta
	% Crystallographica Section D). Repeat for each separate structure e.g
	% \PDBref[dethiobiotin synthetase]{1byi} \NDBref[d(G$_4$CGC$_4$)]{ad0002}
	
	%\PDBref[optional name]{refcode}
	%\NDBref[optional name]{refcode}
	
	%-------------------------------------------------------------------------
	% The introductory (header) part of the paper
	%-------------------------------------------------------------------------
	
	% The title of the paper. Use \shorttitle to indicate an abbreviated title
	% for use in running heads (you will need to uncomment it).
	{\LARGE \emph{\today}} \\
	\title{What are unit cells}
	%\title{Note on the transformation of three-space basis vectors to  corresponding matrix for Delaunay scalars}
	\shorttitle{What are unit cells}
	
	% Authors' names and addresses. Use \cauthor for the main (contact) author.
	% Use \author for all other authors. Use \aff for authors' affiliations.
	% Use lower-case letters in square brackets to link authors to their
	% affiliations; if there is only one affiliation address, remove the [a].
	
	
	\cauthor[a]{Lawrence C.}{Andrews}{larry6640995@gmail.com}{}
	\author[b]{Herbert J.}{Bernstein}
	
	\aff[a]{Ronin Institute, 9515 NE 137th St, Kirkland, WA, 98034-1820 \country{USA}}
	\aff[b]{Ronin Institute, c/o NSLS-II, Brookhaven National Laboratory, Upton, NY, 11973 \country{USA}}
	
	% Use \shortauthor to indicate an abbreviated author list for use in
	% running heads (you will need to uncomment it).
	
	\shortauthor{Andrews and Bernstein}
	
	% Use \vita if required to give biographical details (for authors of
	% invited review papers only). Uncomment it.
	
	% lca IUCr id IUCr6401
	%\vita{Author's biography}
	
	% Keywords (required for Journal of Synchrotron Radiation only)
	% Use the \keyword macro for each word or phrase, e.g. 
	% \keyword{X-ray diffraction}\keyword{muscle}
	
	\keyword{lattice}
	\keyword{unit cells}
	\keyword{Delone}
	\keyword{Selling}
	
	% PDB and NDB reference codes for structures referenced in the article and
	% deposited with the Protein Data Bank and Nucleic Acids Database (Acta
	% Crystallographica Section D). Repeat for each separate structure e.g
	% \PDBref[dethiobiotin synthetase]{1byi} \NDBref[d(G$_4$CGC$_4$)]{ad0002}
	
	%\PDBref[optional name]{refcode}
	%\NDBref[optional name]{refcode}
	
	\maketitle                        % DO NOT DELETE THIS LINE
	
	\begin{synopsis}
		Unit cells
	\end{synopsis}
	\newcommand{\si}[0]{$s_1$}
	\newcommand{\sii}[0]{$s_2$}
	\newcommand{\siii}[0]{$s_3$}
	\newcommand{\siv}[0]{$s_4$}
	\newcommand{\sv}[0]{$s_5$}
	\newcommand{\svi}[0]{$s_6$}
	\newcommand{\Svec} [0] {\{\si, \sii, \siii, \siv, \sv, \svi \}}
	\newcommand{\SvecA} [0] {\{-\si, -\si+\sii, \si+\siii, \si+\sv, \si+\siv, \si+\svi \}}
	
	\newcommand{\OPES}[0]{$E^3toS^6$}
	\newcommand{\OPESS}[0]{$$E^3toS^6$$}
	\newcommand{\MSVI}[0]{$M_{S^{6}}$}
	\newcommand{\MEIII}[0]{$M_{E^{3}}$}
	\newcommand{\Plus}[0]{\mathcal{P}}	
	\newcommand{\Minus}[0]{\mathcal{M}}
	
	\newcommand{\ci}[0]{$c_1$}
	\newcommand{\cii}[0]{$c_2$}
	\newcommand{\ciii}[0]{$c_3$}
	
	
	\begin{abstract}
		Abstract Unit cells
		
		{\bf Note:}  In his later publications, Boris Delaunay used the Russian version of his surname, Delone.\\
		
		
	\end{abstract}
	% Appendices appear after the main body of the text. They are prefixed by
	% a single \appendix declaration, and are then structured just like the
	% body text.
	
	
	\section{Introduction}
	
	

	
	The term ``unit cell' is now the most commonly 
	used term for a smallest repeating unit 
	of a crystal structure.
	It allows us to describe the crystal structure 
	in terms of a simple geometric shape. 
	This makes it easier to visualize the structure 
	and to calculate its properties.
	
	What is a unit cell? Conventionally, it is either thought
	
	

\section{Conventional 3-space methods}

\subsection{Lattices as basis vectors}

The fundamental understanding of crystals begins from the
study of lattices. A typical lattice 
$\Lambda$  in 
$\mathbb{R}^{n}$ thus has the form

\begin{center}
	$\Lambda =  
\begin{Bmatrix}
	\sum_{i=1}^n a_i \mathbf{v}_i \bigg|~{a}_i \in \mathbb{Z}
\end{Bmatrix}
$,
\end{center}

\noindent
where: ($\mathbf{v}_1,...,\mathbf{v}_n)$ is a basis of the lattice. Conventionally, "the unit cell" is defined as the
set of n vectors generated by $a_i = 1$.

In crystallography, we study the lattice in three-dimensional space (n=3 in the above equation). We use 3x3 matrices to create different unit cells by manipulating a vector of 3D vectors. These manipulations are usually described as happening in 3D space, but the vector of 3D vectors is actually a mathematical object with nine parameters in a space that can be described as $R^3xR^3xR^3$, which is a nine-dimensional space.

\subsection{a,b,c,$\alpha$,$\beta$,$\gamma$}
\label{length/angles}

The second conventional representation of unit cells is
2 triplets: 3 edge lengths of the basis vectors and
the 3 angles between them. This is a 6-dimensional 
space: $R^3xR^3$.

The problem with this representation occurs in the cases
where the need is to compare two (or more) unit cells. 
The representation as  $R^3xR^3$ informs us that the
two  $R^3$ are not necessarily commensurate. Indeed,
the edge lengths and the interaxial angles have no
shared measure. (However, see Section \ref{polar}.)

\subsection{Polar coordinates}
\label{polar}

Section \ref{length/angles} alluded to the incommensurate
nature of the two subspace of $R^3xR^3$. 

\citeasnoun{Delone1975} commented on the importance of the "opposite"
scalars in the Bravais tetrahedron. \citeasnoun{Andrews2019b} observed
that the association of those opposites implied the association of
each unit cell edge with the opposite interaxial angle:

\begin{center}
	\label{polarDef}
	($a,\alpha$), ($b,\beta$), ($c, \gamma$).
\end{center}
\noindent
They used that association to create a space $C^3$ (see Section \ref{c3}).
Using the above definition (\ref{polarDef}, we can define a 6-dimensional
space $P^3$. Unlike the issue with the direct use of lengths and angles,
polar coordinates are a well-known 2-dimensional object with the alternative
representation as $(x,y)$, the 2-dimensional plane, where we
can immediately use Euclidean distance measure.

\subsection{Dirac}
\label{dirac}

Dirac cells are defined as one or more primitive unit cell
between a pair of lattice plans. Physicists use Dirac
cell in the computation of energy levels \cite{dirac1930principles}.

\section{Niggli space}

\section{Delone space}

\subsection{\SVI}
\label{s6}

\subsection{\CIII}
\label{c3}

\subsection{Root Invariant}
\label{ri}
% emailed Kurlin and Bright on 8/14/2023 requesting the
% appropriate reference

\section{Wigner-Seitz space}


	
	
	
	
	\section{Notation}
	

	
	\section{Summary}
	
	Summary unit cells
	
	
	
	
	
	
	\section{Availability of code}
	
	The $C^{++}$ ~code for \CIII{} and related 
	software tools is available in github.com, in
	\url{https://github.com/duck10/LatticeRepLib.git}.
	The program CmdToC3 uses the required files.
	
	%\appendix
	
	
	%\section{blah blah blah -- Supplementary Material}
	\ack{{\bf Acknowledgements}}
	
	Careful copy-editing and corrections by Frances C. Bernstein are 
	gratefully acknowledged.
	Our thanks to Jean Jakoncic and Alexei Soares for 
	helpful conversations and access to data and facilities at 
	Brookhaven National Laboratory.
	
	\ack{{\bf Funding information}}      
	
	Funding for this research was provided in part by:  
	US Department of Energy Offices of Biological and 
	Environmental Research and of Basic Energy Sciences 
	(grant No. DE-AC02-98CH10886; grant No. E-SC0012704); 
	U.S. National Institutes of Health (grant No. P41RR012408; 
	grant No. P41GM103473; grant No. P41GM111244; 
	grant No. R01GM117126,
	grant No. 1R21GM129570); Dectris, Ltd.
	
	
	\bibliography{Reduced}
	
	\bibliographystyle{iucr}
	
	
	
	%-------------------------------------------------------------------------
	% TABLES AND FIGURES SHOULD BE INSERTED AFTER THE MAIN BODY OF THE TEXT
	%-------------------------------------------------------------------------
	
	% Simple tables should use the tabular environment according to this
	% model
	
	% Postscript figures can be included with multiple figure blocks
	
	%C:\Users\lca\Source\Repos\LatticeRepLib\x64\Debug>plotc3
	%; Graphical output SVG file =PLT__2023-03-07.13_43_35.svg
	%
	%C:\Users\lca\Source\Repos\LatticeRepLib\x64\Debug>cmdniggli | plotc3
	%; Graphical output SVG file =PLT__2023-03-07.13_44_06.svg
	%
	%C:\Users\lca\Source\Repos\LatticeRepLib\x64\Debug>cmddelone | plotc3
	%; Graphical output SVG file =PLT__2023-03-08.07_11_03.svg
	%
	%C:\Users\lca\Source\Repos\LatticeRepLib\x64\Debug>cmdniggli | cmdperturb 5 20 | plotc3
	%; Graphical output SVG file =PLT__2023-03-08.09_00_13.svg
	%
	%C:\Users\lca\Source\Repos\LatticeRepLib\x64\Debug>cmdniggli | cmdperturb 5 100 | plotc3
	%; Graphical output SVG file =PLT__2023-03-08.09_00_21.svg
	
\end{document}                    % DO NOT DELETE THIS LINE
%%%%%%%%%%%%%%%%%%%%%%%%%%%%%%%%%%%%%%%%%%%%%%%%%%%%%%%%%%%%%%%%%%%%%%%%%%%%%%
