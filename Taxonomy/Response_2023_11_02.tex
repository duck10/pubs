%------------------------------------------------------------------------------
% Template file for the submission of papers to IUCr journals in LaTeX2e
% using the iucr document class
% Copyright 1999-2013 International Union of Crystallography
% Version 1.6 (28 March 2013)
%------------------------------------------------------------------------------

\documentclass[preprint]{iucr}              % DO NOT DELETE THIS LINE
\usepackage{amssymb}
\usepackage[fleqn]{amsmath}
%\usepackage{bm}
\usepackage{graphicx}
\usepackage{tabularx}
\usepackage{booktabs}
%\usepackage{calligra}
\usepackage{array}
\DeclareMathAlphabet{\mathcalligra}{T1}{calligra}{m}{n}
\def\mathbi#1{\textbf{\em #1}}
\numberwithin{equation}{section}
%\DeclareMathSymbol{\Gamma}{\mathalpha}{letters}{"00}
%\DeclareMathSymbol{\Lambda}{\mathalpha}{letters}{"03}
%\DeclareMathSymbol{\Omega}{\mathalpha}{letters}{"0A}
%\DeclareMathAlphabet{\mathitbf}{OML}{cmm}{b}{it}
\hyphenation{Niggli}
\def\mathbi#1{\textbf{\em #1}}
%\numberwithin{equation}{section}
%\DeclareMathSymbol{\Gamma}{\mathalpha}{letters}{"00}
%\DeclareMathSymbol{\Lambda}{\mathalpha}{letters}{"03}
%\DeclareMathSymbol{\Omega}{\mathalpha}{letters}{"0A}
%\DeclareMathAlphabet{\mathitbf}{OML}{cmm}{b}{it}
\DeclareRobustCommand{\captionpar}{\par}
\usepackage{color}
\usepackage{ulem}
\usepackage{url}
\usepackage{yfonts}
\usepackage{relsize} 
\usepackage{wrapfig}
%\usepackage{xr-hyper}
%\usepackage[draft]{hyperref}
%\usepackage{bibentry}

\newcommand{\SVI}[0]{$\bf{S^{6}}$}
\newcommand{\GVI}[0]{$\bf{G^{6}}$}
\newcommand{\CIII}[0]{$\bf{C^{3}}$}
\newcommand{\DVII}[0]{$\bf{D^{7}}$}
\newcommand{\VVII}[0]{$\bf{V^{7}}$}

\newcommand{\vdotv}[2]{${{\bf #1 \cdot #2}}$}
\newcommand{\Imaginary}[0]{$ \textfrak{I} $}
\newcommand{\Real}[0]{$ \textfrak{R} $}
\newcommand{\Exchange}[0]{$\textfrak{X}$}

\newcommand{\nounderline}[3]{\!\!\!\!\!\!\!\!\!#1,&\!\!\!\!\!\!\!\!\!#2,&\!\!\!\!\!\!\!\!\!#3}
\newcommand{\underlineab}[3]{\!\!\!\!\!\!\!\!\!\!\!\!\!\!\!\!\!\!\!\!\!\!\!\!\Exchange{}(#1),&\!\!\!\!\!\!\!\!\!\!\!\!\!\!\!\!\!\!\!\!\!\!\!\!\Exchange{}(#2),&\!\!\!\!\!\!\!\!\!#3}
\newcommand{\underlineac}[3]{\!\!\!\!\!\!\!\!\!\!\!\!\!\!\!\!\!\!\!\!\!\!\!\!\Exchange{}(#1),&\!\!\!\!\!\!\!\!\!\!\!\!\!\!\!\!\!\!\!\!\!\!\!\!#2,&\!\!\!\!\!\!\!\!\!\Exchange{}(#3)}
\newcommand{\underlinebc}[3]{\!\!\!\!\!\!\!\!\!\!\!\!\!\!\!\!\!\!\!\!\!\!\!\!#1,&\!\!\!\!\!\!\!\!\!\Exchange{}(#2),&\!\!\!\!\!\!\!\!\!\!\!\!\!\!\!\!\!\!\!\!\!\!\!\!\Exchange{}(#3)}


\newcommand{\scalar}[1]{$#1$}

\newcommand{\scalarsub}[2]{$#1_#2$}

%-------------------------------------------------------------------------
% Information about journal to which submitted
%-------------------------------------------------------------------------
\journalcode{A}              % Indicate the journal to which submitted
%   A - Acta Crystallographica Section A
%   B - Acta Crystallographica Section B
%   C - Acta Crystallographica Section C
%   D - Acta Crystallographica Section D
%   E - Acta Crystallographica Section E
%   F - Acta Crystallographica Section F
%   J - Journal of Applied Crystallography
%   M - IUCrJ
%   S - Journal of Synchrotron Radiation
\makeatletter
\font\dummyft@=dummy \relax
\makeatother

\usepackage{setspace}


\begin{document}                  % DO NOT DELETE THIS LINE

	\raggedbottom
	\setlength{\parskip}{0pt}
	{\LARGE \emph{\today}} \\
%	\title{Measuring Lattices}
%	%\title{Note on the transformation of three-space basis vectors to  corresponding matrix for Delaunay scalars}
%	\shorttitle{Measuring Lattices}
		
	
	\cauthor[a]{Lawrence C.}{Andrews}{larry6640995@gmail.com}{}
	\author[b]{Herbert J.}{Bernstein}
	
	\aff[a]{Ronin Institute, 9515 NE 137th St, Kirkland, WA, 98034-1820 \country{USA}}
	\aff[b]{Ronin Institute, c/o NSLS-II, Brookhaven National Laboratory, Upton, NY, 11973-5000 \country{USA}}
	
	
	% PDB and NDB reference codes for structures referenced in the article and
	% deposited with the Protein Data Bank and Nucleic Acids Database (Acta
	% Crystallographica Section D). Repeat for each separate structure e.g
	% \PDBref[dethiobiotin synthetase]{1byi} \NDBref[d(G$_4$CGC$_4$)]{ad0002}
	
	%\PDBref[optional name]{refcode}
	%\NDBref[optional name]{refcode}
	
	\maketitle                        % DO NOT DELETE THIS LINE
	\newcommand{\si}[0]{$s_1$}
	\newcommand{\sii}[0]{$s_2$}
	\newcommand{\siii}[0]{$s_3$}
	\newcommand{\siv}[0]{$s_4$}
	\newcommand{\sv}[0]{$s_5$}
	\newcommand{\svi}[0]{$s_6$}
	\newcommand{\Svec} [0] {\{\si, \sii, \siii, \siv, \sv, \svi \}}
	\newcommand{\SvecA} [0] {\{-\si, -\si+\sii, \si+\siii, \si+\sv, \si+\siv, \si+\svi \}}
	
	\newcommand{\OPES}[0]{$E^3toS^6$}
	\newcommand{\OPESS}[0]{$$E^3toS^6$$}
	\newcommand{\MSVI}[0]{$M_{S^{6}}$}
	\newcommand{\MEIII}[0]{$M_{E^{3}}$}
	\newcommand{\Plus}[0]{$\textfrak{P}$}	
	\newcommand{\Minus}[0]{$\textfrak{M}$}
	
	\newcommand{\ci}[0]{$c_1$}
	\newcommand{\cii}[0]{$c_2$}
	\newcommand{\ciii}[0]{$c_3$}
	

	\singlespacing
	
	\section{Note}
	
	Although this document is written in first person, Herbert
	Bernstein has edited and agreed to all statements.

	\section{Response to Reviewer 1}
		
	I would like to begin with the response to the issues of the reviewer
	by quoting the abstract of our article:
	
\begin{quote}
	Several attributions exist for the reduced cells of lattices
	and for the reduction processes to produce them. The actual
	origins of the terms are compared and a taxonomy created. The
	terms ``Niggli reduction", ``Niggli cell", ``Delaunay reduction",
	and ``Delaunay cell" most accurately describe the crystallographic usages
	of reduced cells and reduction methods. 
\end{quote}

It seems that Reviewer 1 has misunderstood the aim of the article, which is clearly given in the abstract.

Reviewer 1:

\begin{quotation}
	Current manuscript suggests brief description of various approaches to the choice of
reduced cells of crystal lattices and associated reduction methods. In fact, the problem of
reduction for lattices has much more general mathematical meaning and consists in finding
conditions on parameters of an “elementary cell” allowing to find for any lattice the unique
(in certain sense) frame characterizing the elementary cell. The application to classification
of 3-dimensional crystal lattices is an important physical problem and it is certainly
very useful for specialists in crystallography to adapt and to follow terminology and
the mathematical results formulated over more than 250-years history of such celebrated
mathematical problem.
After reading the current.
\end{quotation}

We completely agree with the reviewer's statements. But this is not
the paper, nor the intent, to describe the important topics they allude to.
Even our abstract is clear that this is not the intent. If the reviewer
would prepare the article they describe, I would eagerly read it.

On the issue of \cite{Delone1975}: unfortunately neither of us knows
the Russian language, so we have always consulted the English translation.
We agree that that paper is an important one. To point out an
important point from that paper, they discuss unit cells as being in
a metric space; it is the same space that we have elaborated as S6 
in our own work.

The reviewer's comment has led us to include a
reference to that very useful article.


	\section{Response to Reviewer 2}
	
	The reviewer writes:
	
\begin{quotation}
		Ten methods are described in sequence included in alphabetical order of the authors. It is noticeable that the latest
	date for the references of the methods is 1960 and many are from the 19th century. So I doubt the timeliness and
	tbe relevance to crystallographers of this contribution.
\end{quotation}

It is true that the table that lists the types and their taxonomy does
only have dates from 1960 and farther into the past. That is because
the table lists the types and the dates of their original 
creations/publication.

The later text cites dates including 2023. 

The only other issue this reviewer lists is whether to change
the ordering of the descriptions of the types to chronological.
While they could be changed, that change would make locating
they particular types more difficult for someone who is not familiar with the original dates. I do not see the
advantage of such a change.


% \citeasnoun{OishiTomiyasu2012}


			
			\bibliography{Reduced}
			
			\bibliographystyle{iucr}
			
			
			
			%-------------------------------------------------------------------------
			% TABLES AND FIGURES SHOULD BE INSERTED AFTER THE MAIN BODY OF THE TEXT
			%-------------------------------------------------------------------------
			
			% Simple tables should use the tabular environment according to this
			% model
			
			% Postscript figures can be included with multiple figure blocks
			
			
			
		\end{document}                    % DO NOT DELETE THIS LINE
		%%%%%%%%%%%%%%%%%%%%%%%%%%%%%%%%%%%%%%%%%%%%%%%%%%%%%%%%%%%%%%%%%%%%%%%%%%%%%%
