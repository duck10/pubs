%------------------------------------------------------------------------------
% Template file for the submission of papers to IUCr journals in LaTeX2e
% using the iucr document class
% Copyright 1999-2013 International Union of Crystallography
% Version 1.6 (28 March 2013)
%------------------------------------------------------------------------------

%xxxxxx just for testing github % 2
%xxxxxx just for testing github % 3

\documentclass[preprint]{iucr}              % DO NOT DELETE THIS LINE
\usepackage{amssymb}
\usepackage[fleqn]{amsmath}
%\usepackage{bm}
\usepackage{graphicx}
\usepackage{tabularx}
\usepackage{booktabs}
%\usepackage{calligra}
\usepackage{array}
\DeclareMathAlphabet{\mathcalligra}{T1}{calligra}{m}{n}
\def\mathbi#1{\textbf{\em #1}}
\numberwithin{equation}{section}
%\DeclareMathSymbol{\Gamma}{\mathalpha}{letters}{"00}
%\DeclareMathSymbol{\Lambda}{\mathalpha}{letters}{"03}
%\DeclareMathSymbol{\Omega}{\mathalpha}{letters}{"0A}
%\DeclareMathAlphabet{\mathitbf}{OML}{cmm}{b}{it}
\hyphenation{Niggli}
\def\mathbi#1{\textbf{\em #1}}
%\numberwithin{equation}{section}
%\DeclareMathSymbol{\Gamma}{\mathalpha}{letters}{"00}
%\DeclareMathSymbol{\Lambda}{\mathalpha}{letters}{"03}
%\DeclareMathSymbol{\Omega}{\mathalpha}{letters}{"0A}
%\DeclareMathAlphabet{\mathitbf}{OML}{cmm}{b}{it}
\usepackage{color}
%\usepackage{ulem}
\usepackage{url}
\usepackage{yfonts}
%\usepackage{xr-hyper}
%\usepackage[draft]{hyperref}
%\usepackage{bibentry}
\usepackage[normalem]{ulem}
% Redefine \underline to use \textit
\renewcommand{\underline}[1]{\textit{#1}}

% Crystallographic space macros
\newcommand{\HVI}{\ensuremath{\mathbf{H}^{6}}}
\newcommand{\GVI}{\ensuremath{\mathbf{G}^{6}}}
\newcommand{\SVI}{\ensuremath{\mathbf{S}^{6}}}
\newcommand{\PIII}{\ensuremath{\mathbf{P}^{3}}}
\newcommand{\FIII}{\ensuremath{\mathbf{F}^{3}}}
\newcommand{\CIII}{\ensuremath{\mathbf{C}^{3}}}
\newcommand{\RIII}{\ensuremath{\mathbb{R}^{3}}}

\newcommand{\vdotv}[2]{${{\bf #1 \cdot #2}}$}
\newcommand{\Imaginary}[0]{\mathcal{I}}
\newcommand{\Real}[0]{\mathcal{R}}
\newcommand{\Exchange}[0]{$\mathcal{X}$}

\newcommand{\nounderline}[3]{\!\!\!\!\!\!\!\!\!#1,&\!\!\!\!\!\!\!\!\!#2,&\!\!\!\!\!\!\!\!\!#3}
\newcommand{\underlineab}[3]{\!\!\!\!\!\!\!\!\!\!\!\!\!\!\!\!\!\!\!\!\!\!\!\!\Exchange{}(#1),&\!\!\!\!\!\!\!\!\!\!\!\!\!\!\!\!\!\!\!\!\!\!\!\!\Exchange{}(#2),&\!\!\!\!\!\!\!\!\!#3}
\newcommand{\underlineac}[3]{\!\!\!\!\!\!\!\!\!\!\!\!\!\!\!\!\!\!\!\!\!\!\!\!\Exchange{}(#1),&\!\!\!\!\!\!\!\!\!\!\!\!\!\!\!\!\!\!\!\!\!\!\!\!#2,&\!\!\!\!\!\!\!\!\!\Exchange{}(#3)}
\newcommand{\underlinebc}[3]{\!\!\!\!\!\!\!\!\!\!\!\!\!\!\!\!\!\!\!\!\!\!\!\!#1,&\!\!\!\!\!\!\!\!\!\Exchange{}(#2),&\!\!\!\!\!\!\!\!\!\!\!\!\!\!\!\!\!\!\!\!\!\!\!\!\Exchange{}(#3)}

\newcommand{\scalar}[1]{$#1$}

\newcommand{\scalarsub}[2]{$#1_#2$}

%-------------------------------------------------------------------------
% Information about journal to which submitted
%-------------------------------------------------------------------------
\journalcode{A}              % Indicate the journal to which submitted
%   A - Acta Crystallographica Section A
%   B - Acta Crystallographica Section B
%   C - Acta Crystallographica Section C
%   D - Acta Crystallographica Section D
%   E - Acta Crystallographica Section E
%   F - Acta Crystallographica Section F
%   J - Journal of Applied Crystallography
%   M - IUCrJ
%   S - Journal of Synchrotron Radiation
\makeatletter
\font\dummyft@=dummy \relax
\makeatother


\begin{document}                  % DO NOT DELETE THIS LINE
	
	%-------------------------------------------------------------------------
	% The introductory (header) part of the paper
	%-------------------------------------------------------------------------
	
	% The title of the paper. Use \shorttitle to indicate an abbreviated title
	% for use in running heads (you will need to uncomment it).
	
	% Authors' names and addresses. Use \cauthor for the main (contact) author.
	% Use \author for all other authors. Use \aff for authors' affiliations.
	% Use lower-case letters in square brackets to link authors to their
	% affiliations; if there is only one affiliation address, remove the [a].
	
	% Use \vita if required to give biographical details (for authors of
	% invited review papers only). Uncomment it.
	
	% lca IUCr id IUCr6401
	%\vita{Author's biography}
	
	% Keywords (required for Journal of Synchrotron Radiation only)
	% Use the \keyword macro for each word or phrase, e.g. 
	% \keyword{X-ray diffraction}\keyword{muscle}
	
	
	% PDB and NDB reference codes for structures referenced in the article and
	% deposited with the Protein Data Bank and Nucleic Acids Database (Acta
	% Crystallographica Section D). Repeat for each separate structure e.g
	% \PDBref[dethiobiotin synthetase]{1byi} \NDBref[d(G$_4$CGC$_4$)]{ad0002}
	
	%\PDBref[optional name]{refcode}
	%\NDBref[optional name]{refcode}
	
	%-------------------------------------------------------------------------
	% The introductory (header) part of the paper
	%-------------------------------------------------------------------------
	
	% The title of the paper. Use \shorttitle to indicate an abbreviated title
	% for use in running heads (you will need to uncomment it).
	\begin{center}
	{\LARGE \emph{\today}} \\
\end{center}

\title{Approximate Lattice Matching in Three Dimensions}
\shorttitle{3D Lattice Matching}

% Authors' names and addresses. Use \cauthor for the main (contact) author.
% Use \author for all other authors. Use \aff for authors' affiliations.
% Use lower-case letters in square brackets to link authors to their
% affiliations; if there is only one affiliation address, remove the [a].


\cauthor[a]{Lawrence C.}{Andrews}{larry6640995@gmail.com}{}
\author[b]{Herbert J.}{Bernstein}

\aff[a]{Ronin Institute for Independent Scholarship 2.0, USA}
\aff[b]{Ronin Institute for Independent Scholarship 2.0, USA}

% Use \shortauthor to indicate an abbreviated author list for use in
% running heads (you will need to uncomment it).

\shortauthor{Andrews and Bernstein}

	% lca IUCr id IUCr6401
	% HJB IUCr id IUCr6484
	% NKS IUCr ID: IUCr7572
	% lca ORCID  0000-0002-4451-1641
	% HJB ORCID 0000-0002-0517-8532
	% NKS ORCID 0000-0003-2786-6552
	% Use \shortauthor to indicate an abbreviated author list for use in
	% running heads (you will need to uncomment it).
	
	\shortauthor{Andrews and Bernstein}
	
	% Use \vita if required to give biographical details (for authors of
	% invited review papers only). Uncomment it.
	
	% lca IUCr id IUCr6401
	%\vita{Author's biography}
	
	% Keywords (required for Journal of Synchrotron Radiation only)
	% Use the \keyword macro for each word or phrase, e.g. 
	% \keyword{X-ray diffraction}\keyword{muscle}
	
	\keyword{lattice}
	\keyword{reduction}
	\keyword{Niggli}
	\keyword{matching}
	\keyword{\PIII}
	
	% PDB and NDB reference codes for structures referenced in the article and
	% deposited with the Protein Data Bank and Nucleic Acids Database (Acta
	% Crystallographica Section D). Repeat for each separate structure e.g
	% \PDBref[dethiobiotin synthetase]{1byi} \NDBref[d(G$_4$CGC$_4$)]{ad0002}
	
	%\PDBref[optional name]{refcode}
	%\NDBref[optional name]{refcode}
	
	\maketitle                        % DO NOT DELETE THIS LINE
	
	\begin{synopsis}

	\end{synopsis}

	
	\newcommand{\OPES}[0]{$E^3toS^6$}
	\newcommand{\OPESS}[0]{$$E^3toS^6$$}
	\newcommand{\MSVI}[0]{$M_{S^{6}}$}
	\newcommand{\MEIII}[0]{$M_{E^{3}}$}
	\newcommand{\Plus}[0]{\mathcal{P}}	
	\newcommand{\Minus}[0]{\mathcal{M}}
	
	\newcommand{\ci}[0]{$c_1$}
	\newcommand{\cii}[0]{$c_2$}
	\newcommand{\ciii}[0]{$c_3$}
	
	
	\begin{abstract}
Given two unit cells, the problem of determining the best match
of the lattice of one to the lattice of the other can require
large numbers of trial transformations. We present a solution 
that requires only a limited number of trials.
 
		
		{\bf Note:}  In his later publications, Boris Delaunay used the Russian version of his surname, Delone.\\
		
		
	\end{abstract}
	% Appendices appear after the main body of the text. They are prefixed by
	% a single \appendix declaration, and are then structured just like the
	% body text.
	
	
	\section{Introduction}

	A relatively common crystallographic problem is to compare two 
	unit cells and determine whether their lattices are the or
	close to each other. \citeasnoun{flor2016comparison} have described a
	solution for more symmetric crystal families. However, for the general case of triclinic lattices
	a different method is required.
	
	\citeasnoun{andrews2023approximating} described a general solution, using
	the space \SVI{}. While effective, this solution in a space
	based on projections fails to produce the 3D transformations needed
	for reindexing reflection data and for examining the relationships
	of structures.
	
	 Attempts to solve the problem
	by trying a large number of trial transformations can require
	an unknown and unknowable number of trials of transformation. Here we describe
	a straight-forward approach that uses a fixed, modest number
	of trials and that gives multiple possible matches if they are
	appropriate.
	
	\section{Notation}
	
\noindent \PIII{}: Polar coordinate space. It is a compact and smooth alternative to lengths and angles. \PIII{} is useful for comparing unit cells. \PIII{} is defined as:

	\PIII{}: 
		$(|\vec{a}|,\alpha),\ (|\vec{b}|,\beta),\ (|\vec{c}|,\gamma)$\\
	$\Rightarrow$\\
	\hspace*{0.5cm}$(|\vec{a}|\cos\alpha,\, |\vec{a}|\sin\alpha)$\\
	\hspace*{0.5cm}$(|\vec{b}|\cos\beta,\, |\vec{b}|\sin\beta)$\\
	\hspace*{0.5cm}$(|\vec{c}|\cos\gamma,\, |\vec{c}|\sin\gamma)$\\
	also known as $(p_1, p_2, p_3)$






	\section{Lattice matching}
	
	The general solution to matching two lattice has been to 
	test the comparison using a large number of likely transformations.
	Two problems arise: how many trials to make and how to effectively
	measure the difference between two lattices.
	
	To avoid the need to examine an unknown number of possible 
	transformations, we propose an algorithm with the following stages.
	One of the input cells will be designated ``reference'' and the
	other (to which the transformations in stage three will be applied) will be
	designated ``mobile''.
	\begin{enumerate}
		\item Convert from both (possibly) centered lattices to primitive.
		(Apply matrix $M_{CR}$ for reference and $M_{CM}$ for mobile.) \cite{ITC_VolumeA_2016_A}
		\item Niggli reduce both lattices. (Apply matrix $M_{NR}$ for reference and $M_{NM}$ for mobile.) 
		\item Apply a limited set of transformations to mobile, measuring an
		agreement factor for each trial, keeping those that meet
		an acceptable agreement value. This stage creates matrix $M_{MR}$
		that transforms the modified mobile to the modified reference. \cite{Niggli1928} \cite{Andrews2025b} 
		\item Compare the results of each transformation using
		the \PIII{} Euclidean distance between the reference and the mobile 
		cell. \cite{Andrews2025c} 
	\end{enumerate}

	\subsection{Finishing}
	When all of the transformations in stage three have
	been examined, the group of accepted results needs to be processed.
	The group might be empty or might contain several possible valid
	results.  We call this the ``accepted group''.
	
	In each stage, when a transformation is used, the 
	corresponding matrix is saved for later use. The final transformation
	chain to convert mobile (in its original lattice centering) to reference (in its original centering) is
	
	 \ensuremath{M_{CM}*M_{NM}*M_{MR}*M_{NR}^{-1}*M_{CR}^{-1}}. 
	 
	 This matrix
	chain needs to be computed for each member of the accepted group. Of
	course, for a given reference/mobile pair, only $M_{MR}$ is different for accepted trials.
	
	The immediate question is ``how many transformations will
	need to be tested in stage three''. We have chosen to use all
	the unimodular matrices ({\it i.e.} determinant = 1.0), with
	elements -1, 0, +1. (3480 total matrices). Each transformation
	is tested and acceptable results are accumulated to be ``finished''.

	
	\section{Examples}
	
	\subsection{example 1}
{
	\noindent personal communication, \citeasnoun{aroyo2025pc}\\
	More than one acceptable match was found\\
\noindent Test input\\
p 5.17 3.18 7.74 90. 104.5 90.  (reference)\\
p 30.9616 3.1800 8.1608 90.00 171.18 90.00 (mobile)\\

\begin{verbatim}
Result of matching
--- Match 1 ---
Quality: EXCELLENT P3 Distance 0.000)

Transformation Matrix:
[  1.0000,   0.0000,   4.0000]
[  0.0000,   1.0000,   0.0000]
[ -1.0000,   0.0000,  -3.0000]
Matrix determinant: 1.0000
Reference cell:   P     5.170     3.180     7.740    90.000   104.500    90.000
Transformed cell: P     5.170     3.180     7.740    90.000   104.500    90.000
*** SUCCESS: Excellent match within P3-relative threshold ***

--- Match 2 ---
Quality: EXCELLENT (P3 Distance 0.000)

Transformation Matrix:
[ -1.0000,   0.0000,  -4.0000]
[  0.0000,   1.0000,   0.0000]
[  1.0000,   0.0000,   3.0000]
Matrix determinant: 1.0000
Reference cell:   P     5.170     3.180     7.740    90.000   104.500    90.000
Transformed cell: P     5.170     3.180     7.740    90.000   104.500    90.000
*** SUCCESS: Excellent match within P3-relative threshold ***
\end{verbatim}

\subsection{example 2}
Test input:\\
C 12.770 21.235 14.411 136.017 84.071 111.795  (reference)\\
F 33.151 18.241 20.218 83.054 144.781 120.639 (mobile)\\


\begin{verbatim}
=== LATTICE MATCHING RESULTS === ===================================================
Found 1 excellent match
Success threshold: 1.43e-01  (Fixed strict)

--- Match 1 ---
Quality: EXCELLENT (0.000 )
P3 Distance: 0.000
S6 Angle: 0.00
Transformation Matrix:
[  0.0000,  -0.5000,   0.5000]
[  1.0000,   0.5000,   0.5000]
[  0.0000,   0.5000,   0.5000]
Matrix determinant: 0.5000
Reference cell:   C    12.770    21.235    14.411   136.017    84.071   111.795
Transformed cell: C    12.770    21.235    14.411   136.017    84.071   111.795
*** SUCCESS: Excellent match within P3-relative threshold ***
\end{verbatim}


\subsection{example 3}
	\noindent personal communication 
	\cite{simmons2025pc}\\
	A case where more than one accepted match was found.\\
	Input:\\
	p 10.25 10.74 21.08 87.72 75.97 61.53 (reference)\\
	p 10.25 10.74 21.08 78.96 75.97 61.49 (mobile)\\


	
\begin{verbatim}
		--- Match 1 ---
	Quality: EXCELLENT (0.010 )
	P3 Distance: 0.010
	S6 Angle: 0.01
	Transformation Matrix:
	[ -1.0000,   0.0000,   0.0000]
	[ -1.0000,   1.0000,   0.0000]
	[  0.0000,   0.0000,  -1.0000]
	Matrix determinant: 1.0000
	    Reference cell    p    10.25     10.74     21.08     87.72     75.97     61.53 (reference)\\
			Transformed cell: P    10.250    10.739    21.080    87.714    75.970    61.502
	*** SUCCESS: Excellent match within P3-relative threshold ***
	
	--- Match 2 ---
	Quality: EXCELLENT (0.023 )
	P3 Distance: 0.023
	S6 Angle: 0.03
	Transformation Matrix:
	[ -1.0000,   0.0000,   0.0000]
	[  0.0000,  -1.0000,   0.0000]
	[ -1.0000,   0.0000,   1.0000]
Matrix determinant: 1.0000
    	Reference cell    p    10.25     10.74     21.08     87.72     75.97     61.53 (reference)\\
    	Transformed cell: P    10.250    10.740    21.087    87.674    75.893    61.490
*** SUCCESS: Excellent match within P3-relative threshold ***

\end{verbatim}

\subsection{example 4}{
%%	C:\Users\lca\Source\Repos\duck10\LatticeRepLib\x64\Release>cmdgen 1 C5 | cmdperturb 20 | cmdniggli | cmdtocell
%	; To Cell
%	; Enter control variables and input vectors (type 'end' to finish):
%	end
20 slight variations on a primitive cubic cell
were
created, with a=10.0, each Niggli reduced, and lattice-matched.\\

Input:
\begin{verbatim}
	P   7.579   7.579   7.579  90.000  90.000  90.000 (reference)
	P   7.570   7.573   7.574  89.957  89.937  89.958
	P   7.576   7.580   7.581  89.989  89.957  89.964
	P   7.573   7.574   7.576  89.950  89.946  89.994
	P   7.571   7.578   7.583  90.058  90.029  90.062
	P   7.577   7.578   7.578  90.051  90.033  90.051
	P   7.572   7.576   7.578  90.033  90.034  90.084
	P   7.573   7.576   7.580  90.028  90.060  90.020
	P   7.570   7.573   7.580  89.993  89.977  89.931
	P   7.572   7.572   7.573  89.982  89.934  89.936
	P   7.577   7.580   7.588  89.958  89.970  89.996
	P   7.571   7.581   7.581  90.061  90.028  90.059
	P   7.573   7.574   7.584  90.005  90.023  90.065
	P   7.580   7.581   7.581  89.933  89.996  89.968
	P   7.576   7.581   7.586  89.913  89.990  89.956
	P   7.573   7.574   7.581  89.979  89.983  89.974
	P   7.570   7.575   7.57   90.025  90.063  90.066
	P   7.571   7.579   7.581  89.988  89.993  89.942
	P   7.578   7.581   7.588  90.049  90.023  90.002
	P   7.580   7.580   7.585  89.932  89.989  89.988
	P   7.571   7.576   7.577  89.970  89.993  89.943
\end{verbatim}
	
\begin{verbatim}
		and here is the matching results:
	
	Most common transformation matrices:
	[1,0,0,0,1,0,0,0,1]   : 9 mobiles (P3 agreement: 0.016)
	[1,0,0,0,-1,0,0,0,-1] : 3 mobiles (P3 agreement: 0.015)
	[0,1,0,0,0,-1,-1,0,0] : 2 mobiles (P3 agreement: 0.015)
	[1,0,0,0,0,-1,0,1,0]  : 1 mobiles (P3 agreement: 0.011)
	[0,0,1,1,0,0,0,1,0]   : 1 mobiles (P3 agreement:, 0.012)
	[0,0,1,0,-1,0,1,0,0]  : 1 mobiles (P3 agreement: 0.014)
	[0,0,-1,0,1,0,1,0,0]  : 1 mobiles (P3 agreement: 0.012)
	[0,-1,0,-1,0,0,0,0,-1]: 1 mobiles (P3 agreement: 0.009)
	[-1,0,0,0,0,-1,0,-1,0]: 1 mobiles (P3 agreement: 0.017)
	
	=== BEST MATCH DETAILS ===
	Mobile 2: 8.42e-03 (EXCELLENT)
	Matrix: [1 0 0 0 1 0 0 0 1]
	Reference	 P     7.579     7.579     7.579    90.000    90.000    90.000
	Transformed:     7.576     7.580     7.581    89.989    89.957    89.964
	
	Mobile 15: 9.36e-03 (EXCELLENT)
	Matrix: [0 -1 0 -1 0 0 0 0 -1]
	Reference:   P   7.579     7.579     7.579    90.000    90.000    90.000
	Transformed:     7.574     7.573     7.581    89.983    89.979    89.974
	
	Mobile 13: 1.03e-02 (EXCELLENT)
	Matrix: [1 0 0 0 1 0 0 0 1]
	Reference:   P   7.579     7.579     7.579    90.000    90.000    90.000
	Transformed      7.580     7.581     7.581    89.933    89.996    89.968
	
\end{verbatim}

\section{Limitations}

The group of all unimodular matrices is an infinite group. Clearly, by limiting this algorithm to matrices with elements
-1/0/+1, a complete search cannot be done. However as the examples show, in common cases, the algorithm is effective. One limitation is obvious: if one axial length of the reduced cell is longer than the sum
of the other two axial lengths, then -1/0/+1 cannot explore some regions
that are available for shorter axial lengths. For instance, if one axis
is much longer than the others, then in a matrix for that long axis,
the column elements for that axis cannot have more than one element -1/+1.
Similar arguments apply to the case of a very short axis. In order
to treat these more extreme cases using this algorithm, a larger
range of element values would need to be examined. The consequence is
that the number of matrices and thus the number of trials grows rapidly.


	\section{Availability of code}
	
	The code for Lattice Matching and related 
	software tools are available in github.com, in
	\url{https://github.com/duck10/LatticeRepLib.git}.
	The program CmdLMP3 uses the required files.
	
	%\appendix
	
	
	%\section{blah blah blah -- Supplementary Material}
	\ack{Acknowledgements}

Careful copy-editing and corrections by Frances C. Bernstein are 
gratefully acknowledged.

%	Our thanks to Jean Jakoncic and Alexei Soares for 
%	helpful conversations and access to data and facilities at 
%	Brookhaven National Laboratory.
%	
\ack{{\bf Funding information}}      

Funding for this research was provided in part by:  
US Department of Energy Offices of Biological and 
Environmental Research and of Basic Energy Sciences 
(grant No. DE-AC02-98CH10886; grant No. E-SC0012704); 
U.S. National Institutes of Health (grant No. P41RR012408; 
grant No. P41GM103473; grant No. P41GM111244; 
grant No. R01GM117126,
grant No. 1R21GM129570); Dectris, Ltd.

	
\section{Synopsis}

A solution is proposed for the problem of best matching the 
lattices of unit cells that are described in different
presentations.
	
	\bibliography{Reduced}
	
	\bibliographystyle{iucr}
	
	
	
	%-------------------------------------------------------------------------
	% TABLES AND FIGURES SHOULD BE INSERTED AFTER THE MAIN BODY OF THE TEXT
	%-------------------------------------------------------------------------
	
	% Simple tables should use the tabular environment according to this
	% model
	
	% Postscript figures can be included with multiple figure blocks
	
	%C:\Users\lca\Source\Repos\LatticeRepLib\x64\Debug>plotc3
	%; Graphical output SVG file =PLT__2023-03-07.13_43_35.svg
	%
	%C:\Users\lca\Source\Repos\LatticeRepLib\x64\Debug>cmdniggli | plotc3
	%; Graphical output SVG file =PLT__2023-03-07.13_44_06.svg
	%
	%C:\Users\lca\Source\Repos\LatticeRepLib\x64\Debug>cmddelone | plotc3
	%; Graphical output SVG file =PLT__2023-03-08.07_11_03.svg
	%
	%C:\Users\lca\Source\Repos\LatticeRepLib\x64\Debug>cmdniggli | cmdperturb 5 20 | plotc3
	%; Graphical output SVG file =PLT__2023-03-08.09_00_13.svg
	%
	%C:\Users\lca\Source\Repos\LatticeRepLib\x64\Debug>cmdniggli | cmdperturb 5 100 | plotc3
	%; Graphical output SVG file =PLT__2023-03-08.09_00_21.svg
	
\end{document}                    % DO NOT DELETE THIS LINE
%%%%%%%%%%%%%%%%%%%%%%%%%%%%%%%%%%%%%%%%%%%%%%%%%%%%%%%%%%%%%%%%%%%%%%%%%%%%%%
