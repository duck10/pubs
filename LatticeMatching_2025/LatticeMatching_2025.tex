%------------------------------------------------------------------------------
% Template file for the submission of papers to IUCr journals in LaTeX2e
% using the iucr document class
% Copyright 1999-2013 International Union of Crystallography
% Version 1.6 (28 March 2013)
%------------------------------------------------------------------------------

%xxxxxx just for testing github % 2
%xxxxxx just for testing github % 3

\documentclass[preprint]{iucr}              % DO NOT DELETE THIS LINE
\usepackage{amssymb}
\usepackage[fleqn]{amsmath}
%\usepackage{bm}
\usepackage{graphicx}
\usepackage{tabularx}
\usepackage{booktabs}
%\usepackage{calligra}
\usepackage{array}
\DeclareMathAlphabet{\mathcalligra}{T1}{calligra}{m}{n}
\def\mathbi#1{\textbf{\em #1}}
\numberwithin{equation}{section}
%\DeclareMathSymbol{\Gamma}{\mathalpha}{letters}{"00}
%\DeclareMathSymbol{\Lambda}{\mathalpha}{letters}{"03}
%\DeclareMathSymbol{\Omega}{\mathalpha}{letters}{"0A}
%\DeclareMathAlphabet{\mathitbf}{OML}{cmm}{b}{it}
\hyphenation{Niggli}
\def\mathbi#1{\textbf{\em #1}}
%\numberwithin{equation}{section}
%\DeclareMathSymbol{\Gamma}{\mathalpha}{letters}{"00}
%\DeclareMathSymbol{\Lambda}{\mathalpha}{letters}{"03}
%\DeclareMathSymbol{\Omega}{\mathalpha}{letters}{"0A}
%\DeclareMathAlphabet{\mathitbf}{OML}{cmm}{b}{it}
\usepackage{color}
%\usepackage{ulem}
\usepackage{url}
\usepackage{yfonts}
%\usepackage{xr-hyper}
%\usepackage[draft]{hyperref}
%\usepackage{bibentry}
\usepackage[normalem]{ulem}
% Redefine \underline to use \textit
\renewcommand{\underline}[1]{\textit{#1}}

% Crystallographic space macros
\newcommand{\HVI}{\ensuremath{\mathbf{H}^{6}}}
\newcommand{\GVI}{\ensuremath{\mathbf{G}^{6}}}
\newcommand{\SVI}{\ensuremath{\mathbf{S}^{6}}}
\newcommand{\PIII}{\ensuremath{\mathbf{P}^{3}}}
\newcommand{\FIII}{\ensuremath{\mathbf{F}^{3}}}
\newcommand{\CIII}{\ensuremath{\mathbf{C}^{3}}}
\newcommand{\RIII}{\ensuremath{\mathbb{R}^{3}}}

\newcommand{\vdotv}[2]{${{\bf #1 \cdot #2}}$}
\newcommand{\Imaginary}[0]{\mathcal{I}}
\newcommand{\Real}[0]{\mathcal{R}}
\newcommand{\Exchange}[0]{$\mathcal{X}$}

\newcommand{\nounderline}[3]{\!\!\!\!\!\!\!\!\!#1,&\!\!\!\!\!\!\!\!\!#2,&\!\!\!\!\!\!\!\!\!#3}
\newcommand{\underlineab}[3]{\!\!\!\!\!\!\!\!\!\!\!\!\!\!\!\!\!\!\!\!\!\!\!\!\Exchange{}(#1),&\!\!\!\!\!\!\!\!\!\!\!\!\!\!\!\!\!\!\!\!\!\!\!\!\Exchange{}(#2),&\!\!\!\!\!\!\!\!\!#3}
\newcommand{\underlineac}[3]{\!\!\!\!\!\!\!\!\!\!\!\!\!\!\!\!\!\!\!\!\!\!\!\!\Exchange{}(#1),&\!\!\!\!\!\!\!\!\!\!\!\!\!\!\!\!\!\!\!\!\!\!\!\!#2,&\!\!\!\!\!\!\!\!\!\Exchange{}(#3)}
\newcommand{\underlinebc}[3]{\!\!\!\!\!\!\!\!\!\!\!\!\!\!\!\!\!\!\!\!\!\!\!\!#1,&\!\!\!\!\!\!\!\!\!\Exchange{}(#2),&\!\!\!\!\!\!\!\!\!\!\!\!\!\!\!\!\!\!\!\!\!\!\!\!\Exchange{}(#3)}

\newcommand{\scalar}[1]{$#1$}

\newcommand{\scalarsub}[2]{$#1_#2$}

%-------------------------------------------------------------------------
% Information about journal to which submitted
%-------------------------------------------------------------------------
\journalcode{A}              % Indicate the journal to which submitted
%   A - Acta Crystallographica Section A
%   B - Acta Crystallographica Section B
%   C - Acta Crystallographica Section C
%   D - Acta Crystallographica Section D
%   E - Acta Crystallographica Section E
%   F - Acta Crystallographica Section F
%   J - Journal of Applied Crystallography
%   M - IUCrJ
%   S - Journal of Synchrotron Radiation
\makeatletter
\font\dummyft@=dummy \relax
\makeatother


%Andrea Hill
%2:18 AM (4 hours ago)
%to Herbert, me, Mois
%
%Dear Larry and Herbert
%
%Thanks for the submissions uv5038 and uv5039.
%
%Please cite these in the format Authors, Year, Journal, Volume, Note, ie:
%
%Andrews, L. & Bernstein, H. J. (2025a). Acta Cryst. A81. Submitted (uv5038)
%
%Andrews, L. & Bernstein, H. J. (2025b). Acta Cryst. A81. Submitted (uv5039)
%
%You could also give the article code (uv5038/uv5039) in plain text in the citations, and provide the article title for reviewers.
%
%Best wishes, Andrea

\begin{document}                  % DO NOT DELETE THIS LINE
	
	%-------------------------------------------------------------------------
	% The introductory (header) part of the paper
	%-------------------------------------------------------------------------
	
	% The title of the paper. Use \shorttitle to indicate an abbreviated title
	% for use in running heads (you will need to uncomment it).
	
	% Authors' names and addresses. Use \cauthor for the main (contact) author.
	% Use \author for all other authors. Use \aff for authors' affiliations.
	% Use lower-case letters in square brackets to link authors to their
	% affiliations; if there is only one affiliation address, remove the [a].
	
	% Use \vita if required to give biographical details (for authors of
	% invited review papers only). Uncomment it.
	
	% lca IUCr id IUCr6401
	%\vita{Author's biography}
	
	% Keywords (required for Journal of Synchrotron Radiation only)
	% Use the \keyword macro for each word or phrase, e.g. 
	% \keyword{X-ray diffraction}\keyword{muscle}
	
	
	% PDB and NDB reference codes for structures referenced in the article and
	% deposited with the Protein Data Bank and Nucleic Acids Database (Acta
	% Crystallographica Section D). Repeat for each separate structure e.g
	% \PDBref[dethiobiotin synthetase]{1byi} \NDBref[d(G$_4$CGC$_4$)]{ad0002}
	
	%\PDBref[optional name]{refcode}
	%\NDBref[optional name]{refcode}
	
	%-------------------------------------------------------------------------
	% The introductory (header) part of the paper
	%-------------------------------------------------------------------------
	
	% The title of the paper. Use \shorttitle to indicate an abbreviated title
	% for use in running heads (you will need to uncomment it).
	\begin{center}
	{\LARGE \emph{\today}} \\
\end{center}

\title{Approximate Lattice Matching in Three Dimensions}
\shorttitle{3D Lattice Matching}

% Authors' names and addresses. Use \cauthor for the main (contact) author.
% Use \author for all other authors. Use \aff for authors' affiliations.
% Use lower-case letters in square brackets to link authors to their
% affiliations; if there is only one affiliation address, remove the [a].


\cauthor[a]{Lawrence C.}{Andrews}{larry6640995@gmail.com}{}
\author[a]{Herbert J.}{Bernstein}

\aff[a]{Ronin Institute for Independent Scholarship 2.0, USA}


% Use \shortauthor to indicate an abbreviated author list for use in
% running heads (you will need to uncomment it).

\shortauthor{Andrews and Bernstein}

	% lca IUCr id IUCr6401
	% HJB IUCr id IUCr6484
	% NKS IUCr ID: IUCr7572
	% lca ORCID  0000-0002-4451-1641
	% HJB ORCID 0000-0002-0517-8532
	% NKS ORCID 0000-0003-2786-6552
	% Use \shortauthor to indicate an abbreviated author list for use in
	% running heads (you will need to uncomment it).
	
	\shortauthor{Andrews and Bernstein}
	
	% Use \vita if required to give biographical details (for authors of
	% invited review papers only). Uncomment it.
	
	% lca IUCr id IUCr6401
	%\vita{Author's biography}
	
	% Keywords (required for Journal of Synchrotron Radiation only)
	% Use the \keyword macro for each word or phrase, e.g. 
	% \keyword{X-ray diffraction}\keyword{muscle}
	
	\keyword{lattice}
	\keyword{reduction}
	\keyword{Niggli}
	\keyword{matching}
	\keyword{\PIII}
	
	% PDB and NDB reference codes for structures referenced in the article and
	% deposited with the Protein Data Bank and Nucleic Acids Database (Acta
	% Crystallographica Section D). Repeat for each separate structure e.g
	% \PDBref[dethiobiotin synthetase]{1byi} \NDBref[d(G$_4$CGC$_4$)]{ad0002}
	
	%\PDBref[optional name]{refcode}
	%\NDBref[optional name]{refcode}
	
	\maketitle                        % DO NOT DELETE THIS LINE
	
	\newcommand{\OPES}[0]{$E^3toS^6$}
	\newcommand{\OPESS}[0]{$$E^3toS^6$$}
	\newcommand{\MSVI}[0]{$M_{S^{6}}$}
	\newcommand{\MEIII}[0]{$M_{E^{3}}$}
	\newcommand{\Plus}[0]{\mathcal{P}}	
	\newcommand{\Minus}[0]{\mathcal{M}}
	
	\newcommand{\ci}[0]{$c_1$}
	\newcommand{\cii}[0]{$c_2$}
	\newcommand{\ciii}[0]{$c_3$}
	
	
	\begin{abstract}
Given two unit cells, the problem of determining the best match
of the lattice of one to the lattice of the other can require
large numbers of trial transformations. We present a solution 
that requires only a limited number of trials.
		
		
	\end{abstract}
	% Appendices appear after the main body of the text. They are prefixed by
	% a single \appendix declaration, and are then structured just like the
	% body text.
	
	
	\section{Introduction}

	A relatively common crystallographic problem is to compare two 
	unit cells and determine whether their lattices are the same or
	close to each other. \citeasnoun{flor2016comparison}  have described a
	solution (COMPSTRU) for more symmetric crystal families. However, for the general case of monoclinic or triclinic lattices
	a different method is required.
	
	\citeasnoun{andrews2023approximating} described a general solution, using
	the space \SVI{}. While effective, this solution in a space
	based on projections fails to produce the 3D transformations needed
	for reindexing reflection data and for examining the relationships
	of structures.

To fill the need to the working crystallographer for transformations to
reindex existing data or transform structural data to a new coordinate system,
the need is for the 3-space transformations.


	\citeasnoun{Mighell1996} give a flow chart that partly encompasses the
	method here described, but no details are given there. Whereas their
	method relies completely on Niggli reduction, our method does
	not, only including Niggli reduction as part of the process.
	In addition, their description of the generation of alternative
	unit cells is quite brief, focusing on subcell and supercell
	transformations.
	
	 Attempts to solve the problem
	by trying a large number of trial transformations can require
	an unknown and unknowable number of trials of transformations. 
	For closely similar lattices, the Niggli reduced cells are, in general, related by a small number of low-order lattice transformations.  Here we describe
	a straight-forward approach that uses a fixed, modest number
	of trials and that gives multiple possible matches if they are
	near or identical to the best match.
	
	\section{Notation}
	
\noindent \PIII{}: Polar coordinate space. It is a compact and smooth alternative to lengths and angles. \PIII{} is useful for comparing unit cells. \PIII{} is defined as:

	\PIII{}: 
		$(|\vec{a}|,\alpha),\ (|\vec{b}|,\beta),\ (|\vec{c}|,\gamma)$\\
	$\Rightarrow$\\
	\hspace*{0.5cm}$(|\vec{a}|\cos\alpha,\, |\vec{a}|\sin\alpha)$\\
	\hspace*{0.5cm}$(|\vec{b}|\cos\beta,\, |\vec{b}|\sin\beta)$\\
	\hspace*{0.5cm}$(|\vec{c}|\cos\gamma,\, |\vec{c}|\sin\gamma)$\\
	also known as $(p_1, p_2, p_3)$






	\section{Lattice matching}
	
	The general solution approach to matching two lattices has been to 
	test the comparison using a large number of likely transformations.
	Two problems arise: how many trials to make and how to effectively
	measure the difference between two lattices.
	
	To avoid the need to examine an unknown number of possible 
	transformations, we propose an algorithm with the following stages.
	One of the input cells will be designated ``reference'' and the
	other (to which the transformations in stage three will be applied) will be
	designated ``mobile''.
	\begin{enumerate}
		\item Convert from both (possibly) centered lattices to primitive.
		(Apply matrix $M_{CR}$ for reference and $M_{CM}$ for mobile.) \cite{ITC_VolumeA_2016_A}
		\item Niggli reduce both lattices. (Apply matrix $M_{NR}$ for reference and $M_{NM}$ for mobile.) 
		\item Apply a limited set of transformations to mobile, measuring an
		agreement factor for each trial, keeping those that meet
		an acceptable agreement value. This stage creates matrix $M_{MR}$
		that transforms the modified mobile to the modified reference. \cite{Niggli1928}  
		\item Compare the results of each transformation using
		the \PIII{} Euclidean distance between the reference and the mobile 
		cell. \cite{Andrews2025a} 
	\end{enumerate}

	\subsection{Finishing}
	When all of the transformations in stage three have
	been examined, the group of accepted results needs to be processed.
	The group might be empty or might contain several possible valid
	results.  We call this the ``accepted group''.
	
	In each stage, when a transformation is used, the 
	corresponding matrix is saved for later use. The final transformation
	chain to convert mobile (in its original lattice centering) to reference (in its original centering) is
	
	 \ensuremath{M_{CM}*M_{NM}*M_{MR}*M_{NR}^{-1}*M_{CR}^{-1}}. 
	 
	 This matrix
	chain needs to be computed for each member of the accepted group. Of
	course, for a given reference/mobile pair, only $M_{MR}$ is different for accepted trials.
	
	The immediate question is ``how many transformations will
	need to be tested in stage three''. We have chosen to use all
	the unimodular matrices ({\it i.e.} with determinant = 1.0), with
	elements -1, 0, +1. (Direct computation shows that there are 3480 such matrices.) Each transformation
	is tested and acceptable results are accumulated to be ``finished''.

There is an infinite number of unimodular matrices, corresponding to the fact that for a given lattice, there is an infinite number of possible choices of unit cell. However, the mathematical group that describes the unimodular matrices has a small basis set that by iteration can create the complete, infinite set. In principle, that small set (6: identity, x,y,z unit translations, and 2 axis exchanges) would need to be used iteratively to unknown depth to solve the current problem. We have chosen a modest set that is large enough to create a strong algorithm. It includes all of the simple unit translations and lattice vector
exhanges; see Figure \ref{p1}.

\begin{figure}
	\begin{center}
	\includegraphics[width=\columnwidth]{p1.png}
	\label{p1}
	\caption{Graph of (a,alpha) as coordinate p1 of \PIII{} representation \cite{Andrews2025a}
	with the initial point shown in black. The initial point has been transformed by
	the 3480 unimodular matrices described in the text. Many of the
	generated points are hidden by the symmetry of generation.}
\end{center}
\end{figure}

	
	\section{Examples}
	
	\subsection{example 1}
{

	More than one acceptable match was found\\
\noindent Test input\\
p 5.17 3.18 7.74 90. 104.5 90.  (reference)\\
p 30.9616 3.1800 8.1608 90.00 171.18 90.00 (mobile)\\

\begin{verbatim}
Result of matching
--- Match 1 ---
Quality: EXCELLENT P3 Distance 0.000)

Transformation Matrix:
[  1.0000,   0.0000,   4.0000]
[  0.0000,   1.0000,   0.0000]
[ -1.0000,   0.0000,  -3.0000]
Matrix determinant: 1.0000
Reference cell:   P     5.170     3.180     7.740    90.000   104.500    90.000
Transformed cell: P     5.170     3.180     7.740    90.000   104.500    90.000
*** SUCCESS: Excellent match within P3-relative threshold ***

--- Match 2 ---
Quality: EXCELLENT (P3 Distance 0.000)

Transformation Matrix:
[ -1.0000,   0.0000,  -4.0000]
[  0.0000,   1.0000,   0.0000]
[  1.0000,   0.0000,   3.0000]
Matrix determinant: 1.0000
Reference cell:   P     5.170     3.180     7.740    90.000   104.500    90.000
Transformed cell: P     5.170     3.180     7.740    90.000   104.500    90.000
*** SUCCESS: Excellent match within P3-relative threshold ***
\end{verbatim}

\subsection{example 2}
Test input:\\
C 12.770 21.235 14.411 136.017 84.071 111.795  (reference)\\
F 33.151 18.241 20.218 83.054 144.781 120.639 (mobile)\\


\begin{verbatim}
=== LATTICE MATCHING RESULTS === ===================================================
Found 1 excellent match
Success threshold: 1.43e-01  (Fixed strict)

--- Match 1 ---
Quality: EXCELLENT (0.000 )
P3 Distance: 0.000
S6 Angle: 0.00
Transformation Matrix:
[  0.0000,  -0.5000,   0.5000]
[  1.0000,   0.5000,   0.5000]
[  0.0000,   0.5000,   0.5000]
Matrix determinant: 0.5000
Reference cell:   C    12.770    21.235    14.411   136.017    84.071   111.795
Transformed cell: C    12.770    21.235    14.411   136.017    84.071   111.795
*** SUCCESS: Excellent match within P3-relative threshold ***
\end{verbatim}


\subsection{example 3}


	A case where more than one accepted match was found. (Personal communication: \citeasnoun{simmons2025pc}.)\\
	Input:\\
	p 10.25 10.74 21.08 87.72 75.97 61.53 (reference)\\
	p 10.25 10.74 21.08 78.96 75.97 61.49 (mobile)\\


	
\begin{verbatim}
		--- Match 1 ---
	Quality: EXCELLENT (0.010 )
	P3 Distance: 0.010
	S6 Angle: 0.01
	Transformation Matrix:
	[ -1.0000,   0.0000,   0.0000]
	[ -1.0000,   1.0000,   0.0000]
	[  0.0000,   0.0000,  -1.0000]
	Matrix determinant: 1.0000
	    Reference cell    p    10.25     10.74     21.08     87.72     75.97     61.53 (reference)
			Transformed cell: P    10.250    10.739    21.080    87.714    75.970    61.502
	*** SUCCESS: Excellent match within P3-relative threshold ***
	
	--- Match 2 ---
	Quality: EXCELLENT (0.023 )
	P3 Distance: 0.023
	S6 Angle: 0.03
	Transformation Matrix:
	[ -1.0000,   0.0000,   0.0000]
	[  0.0000,  -1.0000,   0.0000]
	[ -1.0000,   0.0000,   1.0000]
Matrix determinant: 1.0000
    	Reference cell    p    10.25     10.74     21.08     87.72     75.97     61.53 (reference)
    	Transformed cell: P    10.250    10.740    21.087    87.674    75.893    61.490
*** SUCCESS: Excellent match within P3-relative threshold ***

\end{verbatim}

\subsection{example 4}{
%%	C:\Users\lca\Source\Repos\duck10\LatticeRepLib\x64\Release>cmdgen 1 C5 | cmdperturb 20 | cmdniggli | cmdtocell
%	; To Cell
%	; Enter control variables and input vectors (type 'end' to finish):
%	end
Twenty slight variations of a primitive cubic unit cell (a = 10.0), expressed in \SVI{}, were generated by adding random orthogonal perturbation vectors to the original \SVI{} representation. Each perturbation had a length of 0.001 times the \SVI{} norm of the cubic cell and was added to the vector [0, 0, 0, –100, –100, –100]. The resulting \SVI{} vectors were then converted back to unit cell parameters.
 The modified vectors
were lattice-matched to the reference, which is the first entry.\\

Input:
\begin{verbatim}
P    10.000    10.000    10.000    90.000    90.000    90.000 Reference
P     9.994     9.992     9.988    89.958    89.937    89.957
P    10.002    10.003     9.996    89.957    90.036    90.011
P     9.996     9.992     9.993    89.994    89.950    89.946
P     9.990     9.999    10.005    90.058    89.971    89.938
P     9.997     9.999     9.998    90.051    89.949    89.967
P     9.996     9.998     9.990    89.966    89.916    90.033
P    10.002     9.993     9.996    89.980    90.028    89.940
P    10.002     9.992     9.988    89.931    89.977    89.993
P     9.991     9.992     9.991    89.934    89.936    89.982
P    10.002     9.998    10.012    90.030    90.042    89.996
P    10.003     9.990    10.002    89.972    90.061    89.941
P     9.993    10.006     9.992    90.023    89.935    89.995
P    10.002    10.002    10.002    89.968    90.004    90.067
P    10.003    10.010     9.996    90.010    89.956    90.087
P     9.992    10.003     9.994    89.979    89.974    89.983
P     9.999     9.995     9.988    89.934    89.937    90.025
P    10.003     9.990    10.000    89.942    90.012    90.007
P    10.012     9.998    10.002    90.002    90.049    90.023
P    10.008    10.002    10.002    90.012    90.068    89.989
P     9.989     9.996     9.998    89.970    89.993    89.943
\end{verbatim}
	
\begin{verbatim}
		and here are the matching results:
	
Most common transformation matrices:
[-1,0,0,0,-1,0,0,0,1] : 3 mobiles (EXCELLENT, 0.019)
[1,0,0,0,-1,0,0,0,-1] : 2 mobiles (EXCELLENT, 0.013)
[0,0,-1,0,-1,0,-1,0,0]: 2 mobiles (EXCELLENT, 0.022)
[0,-1,0,-1,0,0,0,0,-1]: 2 mobiles (EXCELLENT, 0.020)
[-1,0,0,0,1,0,0,0,-1] : 2 mobiles (EXCELLENT, 0.014)
[-1,0,0,0,0,-1,0,-1,0]: 2 mobiles (EXCELLENT, 0.022)
[1,0,0,0,1,0,0,0,1]   : 1 mobiles (EXCELLENT, 0.016)
[0,1,0,1,0,0,0,0,-1]  : 1 mobiles (EXCELLENT, 0.015)
[0,1,0,0,0,1,1,0,0]   : 1 mobiles (EXCELLENT, 0.017)
[0,1,0,-1,0,0,0,0,1]  : 1 mobiles (EXCELLENT, 0.016)

=== BEST MATCH DETAILS ===
Mobile 2: 1.13e-02 (EXCELLENT)
Matrix: [0 0 -1 1 0 0 0 -1 0]
Transformed:     9.996    10.002    10.003    89.989    89.957    89.964

Mobile 15: 1.23e-02 (EXCELLENT)
Matrix: [-1 0 0 0 0 -1 0 -1 0]
Transformed:     9.992     9.994    10.003    89.979    89.983    89.974

Mobile 13: 1.34e-02 (EXCELLENT)
Matrix: [1 0 0 0 -1 0 0 0 -1]
Transformed:    10.002    10.002    10.002    89.968    89.996    89.933

	
\end{verbatim}

\subsection{Example 5}

Two reported polymorphic forms of triphenylphosphine oxide
hemihydrate, C\textsubscript{18}H\textsubscript{15} PO$\cdot$0.5H\textsubscript{2}O. \cite{mighell2011role}


\begin{verbatim}
	F 19.794 32.54 9.459 90 90  90
	C 9.4313 32.193 10.8435 90 115.742 90
	end
	
	=== PROCESSING INPUT LIST ===
	Total results before processing: 500
	NC Distance: 6.44756
	Range: 4.34e-01 to 6.61e+01 
	Reference P3 norm: 39.244 
	Method: Fixed strict
	Thresholds:
	EXCELLENT <= 1.96e-01 (0.5% of P3)
	GOOD      <= 7.85e-01 (2.0% of P3)
	POOR      <= 3.14e+00 (8.0% of P3)

	
	=== LATTICE MATCHING RESULTS === ===================================================
	Found 2 good matches
	Success threshold: 1.96e-01  (Fixed strict)
	
	--- Match 1 ---
	Quality: GOOD (0.434 )
	P3 Distance: 0.434
	S6 Angle: 0.12 degrees
	Transformation Matrix:
	[ -1.0000,   0.0000,  -2.0000]
	[  0.0000,  -1.0000,   0.0000]
	[ -1.0000,   0.0000,   0.0000]
	Matrix determinant: 2.0000
	Transformed cell: F    19.535    32.193     9.431    90.000    89.964    90.000
	*** SUCCESS: Good match within P3-relative threshold ***
	
	--- Match 2 ---
	Quality: GOOD (0.434 )
	P3 Distance: 0.434
	S6 Angle: 0.12 degrees
	Transformation Matrix:
	[  1.0000,   0.0000,   2.0000]
	[  0.0000,  -1.0000,   0.0000]
	[  1.0000,   0.0000,   0.0000]
	Matrix determinant: 2.0000
	Transformed cell: F    19.535    32.193     9.431    90.000    89.964    90.000
	*** SUCCESS: Good match within P3-relative threshold ***
	
\end{verbatim}

\section{Limitations}

The group of all unimodular matrices is an infinite group. Clearly, by limiting this algorithm to matrices with elements
-1/0/+1, a complete search cannot be done. However as the examples show, in common cases, the algorithm is effective. One limitation is obvious: if one axial length of the reduced cell is longer than the sum
of the other two axial lengths, then -1/0/+1 cannot explore some regions
that are available for shorter axial lengths. For instance, if one axis
is much longer than the others, then in a matrix for that long axis,
the column elements for that axis cannot have more than one element -1/+1.
Similar arguments apply to the case of a very short axis. In order
to treat these more extreme cases using this algorithm, a larger
range of element values would need to be examined. The consequence is
that the number of matrices and thus the number of trials grows rapidly.

	\section{Summary}
Lattice matching based on unit cell parameters and 3-dimensional
transformation matrices is described, including the production
of the matrices for transforming unit cell base vectors from
a probe's to those of a reference cell.

The described algorithm (CmdLMP3) on a modest laptop computer
matched 100 cases in 5 seconds.


	\section{Availability of code}
	
	The code for Lattice Matching (CmdLMP3) and related 
	software tools is available in github.com, in
	\url{https://github.com/duck10/LatticeRepLib.git}.
	The program CmdLMP3 uses the required files.
	
	%\appendix
	
	
	%\section{blah blah blah -- Supplementary Material}
	
\ack{{\bf Acknowledgements}}

Careful copy-editing and corrections by Frances C. Bernstein are 
gratefully acknowledged. M.I. Aroyo kindly provided test examples,  \citeasnoun{aroyo2025pc}\\.

\ack{{\bf Funding information}}      

Funding for this research was provided in part by:  
US Department of Energy Offices of Biological and 
Environmental Research and of Basic Energy Sciences 
(contract No. DE-AC02-98CH10886; contract No. E-SC0012704); 
U.S. National Institutes of Health (grant No. P41RR012408; 
grant No. P41GM103473; grant No. P41GM111244; 
grant No. R01GM117126,
grant No. 1R21GM129570); Dectris, Ltd.

	
\section{Synopsis}

A solution is proposed for the problem of best matching the 
lattices of unit cells that are described in different
presentations along with the 3D matrix
that defines the relationship between matched
lattices.
	
	\bibliography{Reduced}
	
	\bibliographystyle{iucr}
	
	
	
	%-------------------------------------------------------------------------
	% TABLES AND FIGURES SHOULD BE INSERTED AFTER THE MAIN BODY OF THE TEXT
	%-------------------------------------------------------------------------
	
	% Simple tables should use the tabular environment according to this
	% model
	
	% Postscript figures can be included with multiple figure blocks
	
	%C:\Users\lca\Source\Repos\LatticeRepLib\x64\Debug>plotc3
	%; Graphical output SVG file =PLT__2023-03-07.13_43_35.svg
	%
	%C:\Users\lca\Source\Repos\LatticeRepLib\x64\Debug>cmdniggli | plotc3
	%; Graphical output SVG file =PLT__2023-03-07.13_44_06.svg
	%
	%C:\Users\lca\Source\Repos\LatticeRepLib\x64\Debug>cmddelone | plotc3
	%; Graphical output SVG file =PLT__2023-03-08.07_11_03.svg
	%
	%C:\Users\lca\Source\Repos\LatticeRepLib\x64\Debug>cmdniggli | cmdperturb 5 20 | plotc3
	%; Graphical output SVG file =PLT__2023-03-08.09_00_13.svg
	%
	%C:\Users\lca\Source\Repos\LatticeRepLib\x64\Debug>cmdniggli | cmdperturb 5 100 | plotc3
	%; Graphical output SVG file =PLT__2023-03-08.09_00_21.svg
	
\end{document}                    % DO NOT DELETE THIS LINE
%%%%%%%%%%%%%%%%%%%%%%%%%%%%%%%%%%%%%%%%%%%%%%%%%%%%%%%%%%%%%%%%%%%%%%%%%%%%%%
