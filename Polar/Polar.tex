%------------------------------------------------------------------------------
% Template file for the submission of papers to IUCr journals in LaTeX2e
% using the iucr document class
% Copyright 1999-2013 International Union of Crystallography
% Version 1.6 (28 March 2013)
%------------------------------------------------------------------------------

%xxxxxx just for testing github % 2
%xxxxxx just for testing github % 3

\documentclass[]{iucr}              % DO NOT DELETE THIS LINE
\usepackage{amssymb}
\usepackage[fleqn]{amsmath}
%\usepackage{bm}
\usepackage{graphicx}
\usepackage{tabularx}
\usepackage{booktabs}
%\usepackage{calligra}
\usepackage{array}
\DeclareMathAlphabet{\mathcalligra}{T1}{calligra}{m}{n}
\def\mathbi#1{\textbf{\em #1}}
\numberwithin{equation}{section}
%\DeclareMathSymbol{\Gamma}{\mathalpha}{letters}{"00}
%\DeclareMathSymbol{\Lambda}{\mathalpha}{letters}{"03}
%\DeclareMathSymbol{\Omega}{\mathalpha}{letters}{"0A}
%\DeclareMathAlphabet{\mathitbf}{OML}{cmm}{b}{it}
\hyphenation{Niggli}
\def\mathbi#1{\textbf{\em #1}}
%\numberwithin{equation}{section}
%\DeclareMathSymbol{\Gamma}{\mathalpha}{letters}{"00}
%\DeclareMathSymbol{\Lambda}{\mathalpha}{letters}{"03}
%\DeclareMathSymbol{\Omega}{\mathalpha}{letters}{"0A}
%\DeclareMathAlphabet{\mathitbf}{OML}{cmm}{b}{it}
\usepackage{color}
\usepackage{ulem}
\usepackage{url}
\usepackage{yfonts}
%\usepackage{xr-hyper}
%\usepackage[draft]{hyperref}
%\usepackage{bibentry}
\newcommand{\SVI}[0]{$\bf{S^{6}}$}
\newcommand{\GVI}[0]{$\bf{G^{6}}$}
\newcommand{\CIII}[0]{$\bf{C^{3}}$}
\newcommand{\DVII}[0]{$\bf{D^{7}}$}
\newcommand{\VVII}[0]{$\bf{V^{7}}$}

\newcommand{\vdotv}[2]{${{\bf #1 \cdot #2}}$}
\newcommand{\Imaginary}[0]{$ \textfrak{I} $}
\newcommand{\Real}[0]{$ \textfrak{R} $}
\newcommand{\Exchange}[0]{$\textfrak{X}$}

\newcommand{\nounderline}[3]{\!\!\!\!\!\!\!\!\!#1,&\!\!\!\!\!\!\!\!\!#2,&\!\!\!\!\!\!\!\!\!#3}
\newcommand{\underlineab}[3]{\!\!\!\!\!\!\!\!\!\!\!\!\!\!\!\!\!\!\!\!\!\!\!\!\Exchange{}(#1),&\!\!\!\!\!\!\!\!\!\!\!\!\!\!\!\!\!\!\!\!\!\!\!\!\Exchange{}(#2),&\!\!\!\!\!\!\!\!\!#3}
\newcommand{\underlineac}[3]{\!\!\!\!\!\!\!\!\!\!\!\!\!\!\!\!\!\!\!\!\!\!\!\!\Exchange{}(#1),&\!\!\!\!\!\!\!\!\!\!\!\!\!\!\!\!\!\!\!\!\!\!\!\!#2,&\!\!\!\!\!\!\!\!\!\Exchange{}(#3)}
\newcommand{\underlinebc}[3]{\!\!\!\!\!\!\!\!\!\!\!\!\!\!\!\!\!\!\!\!\!\!\!\!#1,&\!\!\!\!\!\!\!\!\!\Exchange{}(#2),&\!\!\!\!\!\!\!\!\!\!\!\!\!\!\!\!\!\!\!\!\!\!\!\!\Exchange{}(#3)}

\newcommand{\scalar}[1]{$#1$}

\newcommand{\scalarsub}[2]{$#1_#2$}
	\newcommand{\si}[0]{$s_1$}
\newcommand{\sii}[0]{$s_2$}
\newcommand{\siii}[0]{$s_3$}
\newcommand{\siv}[0]{$s_4$}
\newcommand{\sv}[0]{$s_5$}
\newcommand{\svi}[0]{$s_6$}
\newcommand{\Svec} [0] {\{\si, \sii, \siii, \siv, \sv, \svi \}}
\newcommand{\SvecA} [0] {\{-\si, -\si+\sii, \si+\siii, \si+\sv, \si+\siv, \si+\svi \}}

\newcommand{\OPES}[0]{$E^3toS^6$}
\newcommand{\OPESS}[0]{$$E^3toS^6$$}
\newcommand{\MSVI}[0]{$M_{S^{6}}$}
\newcommand{\MEIII}[0]{$M_{E^{3}}$}
\newcommand{\Plus}[0]{$\textfrak{P}$}	
\newcommand{\Minus}[0]{$\textfrak{M}$}

\newcommand{\ci}[0]{$c_1$}
\newcommand{\cii}[0]{$c_2$}
\newcommand{\ciii}[0]{$c_3$}

%-------------------------------------------------------------------------
% Information about journal to which submitted
%-------------------------------------------------------------------------
\journalcode{A}              % Indicate the journal to which submitted
%   A - Acta Crystallographica Section A
%   B - Acta Crystallographica Section B
%   C - Acta Crystallographica Section C
%   D - Acta Crystallographica Section D
%   E - Acta Crystallographica Section E
%   F - Acta Crystallographica Section F
%   J - Journal of Applied Crystallography
%   M - IUCrJ
%   S - Journal of Synchrotron Radiation
\makeatletter
\font\dummyft@=dummy \relax
\makeatother


\begin{document}                  % DO NOT DELETE THIS LINE
	
	%-------------------------------------------------------------------------
	% The introductory (header) part of the paper
	%-------------------------------------------------------------------------
	
	% The title of the paper. Use \shorttitle to indicate an abbreviated title
	% for use in running heads (you will need to uncomment it).
	
	% Authors' names and addresses. Use \cauthor for the main (contact) author.
	% Use \author for all other authors. Use \aff for authors' affiliations.
	% Use lower-case letters in square brackets to link authors to their
	% affiliations; if there is only one affiliation address, remove the [a].
	
	% Use \vita if required to give biographical details (for authors of
	% invited review papers only). Uncomment it.
	
	% lca IUCr id IUCr6401
	%\vita{Author's biography}
	
	% Keywords (required for Journal of Synchrotron Radiation only)
	% Use the \keyword macro for each word or phrase, e.g. 
	% \keyword{X-ray diffraction}\keyword{muscle}
	
	
	% PDB and NDB reference codes for structures referenced in the article and
	% deposited with the Protein Data Bank and Nucleic Acids Database (Acta
	% Crystallographica Section D). Repeat for each separate structure e.g
	% \PDBref[dethiobiotin synthetase]{1byi} \NDBref[d(G$_4$CGC$_4$)]{ad0002}
	
	%\PDBref[optional name]{refcode}
	%\NDBref[optional name]{refcode}
	
	%-------------------------------------------------------------------------
	% The introductory (header) part of the paper
	%-------------------------------------------------------------------------
	
	% The title of the paper. Use \shorttitle to indicate an abbreviated title
	% for use in running heads (you will need to uncomment it).
	{\LARGE \emph{\today}} \\
	\title{Unit cells considered in polar coordinates}
	%\title{Note on the transformation of three-space basis vectors to  corresponding matrix for Delaunay scalars}
	\shorttitle{properties of C3}
	
	% Authors' names and addresses. Use \cauthor for the main (contact) author.
	% Use \author for all other authors. Use \aff for authors' affiliations.
	% Use lower-case letters in square brackets to link authors to their
	% affiliations; if there is only one affiliation address, remove the [a].
	
	
	\cauthor[a]{Lawrence C.}{Andrews}{lawrence.andrews@ronininstitute.org}{}
	\author[b]{Herbert J.}{Bernstein}
	
	\aff[a]{Ronin Institute, 9515 NE 137th St, Kirkland, WA, 98034-1820 \country{USA}}
	\aff[b]{Ronin Institute, c/o NSLS-II, Brookhaven National Laboratory, Upton, NY, 11973 \country{USA}}
	
	% Use \shortauthor to indicate an abbreviated author list for use in
	% running heads (you will need to uncomment it).
	
	\shortauthor{Andrews and Bernstein}
	
	% Use \vita if required to give biographical details (for authors of
	% invited review papers only). Uncomment it.
	
	% lca IUCr id IUCr6401
	%\vita{Author's biography}
	
	% Keywords (required for Journal of Synchrotron Radiation only)
	% Use the \keyword macro for each word or phrase, e.g. 
	% \keyword{X-ray diffraction}\keyword{muscle}
	
	\keyword{lattice}
	\keyword{reduction}
	\keyword{Delone}
	\keyword{Selling}
	\keyword{\CIII}
	
	% PDB and NDB reference codes for structures referenced in the article and
	% deposited with the Protein Data Bank and Nucleic Acids Database (Acta
	% Crystallographica Section D). Repeat for each separate structure e.g
	% \PDBref[dethiobiotin synthetase]{1byi} \NDBref[d(G$_4$CGC$_4$)]{ad0002}
	
	%\PDBref[optional name]{refcode}
	%\NDBref[optional name]{refcode}
	
	\maketitle                        % DO NOT DELETE THIS LINE
	
	\begin{synopsis}
		The space \CIII{} is explained in more detail than
		in the original description. Boundary transformations
		of the fundamental unit are described in detail. 
		A graphical presentation of the basic coordinates
		is described and illustrated.
	\end{synopsis}

	
	\begin{abstract}
		abstract
	\end{abstract}

	
	
	\section{Introduction}
	
	The representation of the crystal lattice parameters as
	 $a$, $b$, $c$,	$\alpha$, $\beta$, and $\gamma$ dates
	 from early in the 20th century. The association that $a$ 
	 is opposite to $\alpha$, etc. is at least that old.
	 
	 \citeasnoun{Delone1975} emphasized the relationship of the
	 "opposite" Selling scalars in the Bravais tetrahedron 
	 representation of lattice.
	 
	 \citeasnoun{Andrews2019b} took the association one
	 step farther, combining the "opposite" pairs of the 6 Selling
	 scalars into the 3 complex coordinates. 
	 
	 The above ideas are here carried to another representation
	 of lattice parameters. Taking the concept that the dimension $a$ is 
	 related to the angle $\alpha$, etc. those pairs are considered
	 to be 3 points represented in polar coordinates: 
	 ($a$,$\alpha$), ($b$,$\beta$), and ($c$,$\gamma$).
	 
	 
	
	
	\section{Notation}

	zzzzzzzzzzzzzzzzz
	
%\begin{table}
%	\begin{tabular}{l c c c c c c c c c}
%		\toprule
%		\midrule
%	\bottomrule
%\end{tabular}	\\
%\caption{caption}
%\label{PLA2}
%\end{table}



%\begin{figure}
%\end{figure}
%	
%\begin{figure}
%\end{figure}
%
%\begin{figure}
%\end{figure}
%
%\begin{figure}
%\end{figure}
%
%\begin{figure}
%\end{figure}



	\section{Summary}
	
	xxxxxxxxxxxxxxxxxx
	
	
	
	
	\section{Availability of code}
	
	The $C^{++}$ ~code for \CIII{} and related 
	software tools is available in github.com, in
	\url{https://github.com/duck10/LatticeRepLib.git}.
	The program Radial uses the required files.
	
	%\appendix
	
	
	%\section{blah blah blah -- Supplementary Material}
	\ack{{\bf Acknowledgements}}
	
	Careful copy-editing and corrections by Frances C. Bernstein are 
	gratefully acknowledged.
	Our thanks to Jean Jakoncic and Alexei Soares for 
	helpful conversations and access to data and facilities at 
	Brookhaven National Laboratory.
	
	\ack{{\bf Funding information}}      
	
	Funding for this research was provided in part by:  
	US Department of Energy Offices of Biological and 
	Environmental Research and of Basic Energy Sciences 
	(grant No. DE-AC02-98CH10886; grant No. E-SC0012704); 
	U.S. National Institutes of Health (grant No. P41RR012408; 
	grant No. P41GM103473; grant No. P41GM111244; 
	grant No. R01GM117126,
	grant No. 1R21GM129570); Dectris, Ltd.
	
	
	\bibliography{Reduced}
	
	\bibliographystyle{iucr}
	
	
	
	%-------------------------------------------------------------------------
	% TABLES AND FIGURES SHOULD BE INSERTED AFTER THE MAIN BODY OF THE TEXT
	%-------------------------------------------------------------------------
	
	% Simple tables should use the tabular environment according to this
	% model
	
	% Postscript figures can be included with multiple figure blocks
	
%C:\Users\lca\Source\Repos\LatticeRepLib\x64\Debug>plotc3
%; Graphical output SVG file =PLT__2023-03-07.13_43_35.svg

\end{document}                    % DO NOT DELETE THIS LINE
%%%%%%%%%%%%%%%%%%%%%%%%%%%%%%%%%%%%%%%%%%%%%%%%%%%%%%%%%%%%%%%%%%%%%%%%%%%%%%
