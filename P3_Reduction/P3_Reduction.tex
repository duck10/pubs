%------------------------------------------------------------------------------
% Template file for the submission of papers to IUCr journals in LaTeX2e
% using the iucr document class
% Copyright 1999-2013 International Union of Crystallography
% Version 1.6 (28 March 2013)
%------------------------------------------------------------------------------

\documentclass[preprint]{iucr}              % DO NOT DELETE THIS LINE
\usepackage{amssymb}
\usepackage[fleqn]{amsmath}
%\usepackage{bm}
\usepackage{graphicx}
\usepackage{tabularx}
\usepackage{booktabs}
%\usepackage{calligra}
\usepackage{longtable}
\usepackage{array}
\DeclareMathAlphabet{\mathcalligra}{T1}{calligra}{m}{n}
\def\mathbi#1{\textbf{\em #1}}
\numberwithin{equation}{section}
%\DeclareMathSymbol{\Gamma}{\mathalpha}{letters}{"00}
%\DeclareMathSymbol{\Lambda}{\mathalpha}{letters}{"03}
%\DeclareMathSymbol{\Omega}{\mathalpha}{letters}{"0A}
%\DeclareMathAlphabet{\mathitbf}{OML}{cmm}{b}{it}
\hyphenation{Niggli}
\def\mathbi#1{\textbf{\em #1}}
%\numberwithin{equation}{section}
%\DeclareMathSymbol{\Gamma}{\mathalpha}{letters}{"00}
%\DeclareMathSymbol{\Lambda}{\mathalpha}{letters}{"03}
%\DeclareMathSymbol{\Omega}{\mathalpha}{letters}{"0A}
%\DeclareMathAlphabet{\mathitbf}{OML}{cmm}{b}{it}
\usepackage{color}
\usepackage{ulem}
\usepackage{url}
\usepackage{yfonts}
%\usepackage{xr-hyper}
%\usepackage[draft]{hyperref}
%\usepackage{bibentry}

\usepackage{tikz}
\usetikzlibrary{arrows.meta, positioning}
\usepackage[most]{tcolorbox}

% lca IUCr id IUCr6401
% HJB IUCr id IUCr6484
% NKS IUCr ID: IUCr7572
% lca ORCID  0000-0002-4451-1641
% HJB ORCID 0000-0002-0517-8532
% NKS ORCID 0000-0003-2786-6552


% Space macros
\usepackage{amsmath, amssymb}  % If not already included

% Crystallographic space macros
\newcommand{\HVI}{\ensuremath{\mathbf{H}^{6}}}
\newcommand{\GVI}{\ensuremath{\mathbf{G}^{6}}}
\newcommand{\SVI}{\ensuremath{\mathbf{S}^{6}}}
\newcommand{\PIII}{\ensuremath{\mathbf{P}^{3}}}
\newcommand{\FIII}{\ensuremath{\mathbf{F}^{3}}}
\newcommand{\BIV}{\ensuremath{\mathbf{B}^{4}}}
\newcommand{\CIII}{\ensuremath{\mathbf{C}^{3}}}
\newcommand{\RIII}{\ensuremath{\mathbb{R}^{3}}}

% a,b,c as base vectors of unit cells
\newcommand{\va}{\ensuremath{\mathbf{a}}}
\newcommand{\vb}{\ensuremath{\mathbf{b}}}
\newcommand{\vc}{\ensuremath{\mathbf{c}}}
\newcommand{\vd}{\ensuremath{\mathbf{d}}}



\newcommand{\vdotv}[2]{${{\bf #1 \cdot #2}}$}
\newcommand{\Imaginary}[0]{\mathcal{I}}
\newcommand{\Real}[0]{\mathcal{R}}
\newcommand{\Exchange}{\ensuremath{\mathcal{X}}}


\newcommand{\nounderline}[3]{\!\!\!\!\!\!\!\!\!#1,&\!\!\!\!\!\!\!\!\!#2,&\!\!\!\!\!\!\!\!\!#3}
\newcommand{\underlineab}[3]{\!\!\!\!\!\!\!\!\!\!\!\!\!\!\!\!\!\!\!\!\!\!\!\!\Exchange{}(#1),&\!\!\!\!\!\!\!\!\!\!\!\!\!\!\!\!\!\!\!\!\!\!\!\!\Exchange{}(#2),&\!\!\!\!\!\!\!\!\!#3}
\newcommand{\underlineac}[3]{\!\!\!\!\!\!\!\!\!\!\!\!\!\!\!\!\!\!\!\!\!\!\!\!\Exchange{}(#1),&\!\!\!\!\!\!\!\!\!\!\!\!\!\!\!\!\!\!\!\!\!\!\!\!#2,&\!\!\!\!\!\!\!\!\!\Exchange{}(#3)}
\newcommand{\underlinebc}[3]{\!\!\!\!\!\!\!\!\!\!\!\!\!\!\!\!\!\!\!\!\!\!\!\!#1,&\!\!\!\!\!\!\!\!\!\Exchange{}(#2),&\!\!\!\!\!\!\!\!\!\!\!\!\!\!\!\!\!\!\!\!\!\!\!\!\Exchange{}(#3)}

\newcommand{\scalar}[1]{\ensuremath{#1}}
\newcommand{\scalarsub}[2]{\ensuremath{#1_{#2}}}


\usepackage{makecell}
\renewcommand\cellalign{l}
\renewcommand\cellgape{\Gape[4pt]}


%-------------------------------------------------------------------------
% Information about journal to which submitted
%-------------------------------------------------------------------------
\journalcode{A}              % Indicate the journal to which submitted
%   A - Acta Crystallographica Section A
%   B - Acta Crystallographica Section B
%   C - Acta Crystallographica Section C
%   D - Acta Crystallographica Section D
%   E - Acta Crystallographica Section E
%   F - Acta Crystallographica Section F
%   J - Journal of Applied Crystallography
%   M - IUCrJ
%   S - Journal of Synchrotron Radiation
\makeatletter
\font\dummyft@=dummy \relax
\makeatother


\begin{document}                  % DO NOT DELETE THIS LINE
	
	%-------------------------------------------------------------------------
	% The introductory (header) part of the paper
	%-------------------------------------------------------------------------
	
	% The title of the paper. Use \shorttitle to indicate an abbreviated title
	% for use in running heads (you will need to uncomment it).
	
	% Authors' names and addresses. Use \cauthor for the main (contact) author.
	% Use \author for all other authors. Use \aff for authors' affiliations.
	% Use lower-case letters in square brackets to link authors to their
	% affiliations; if there is only one affiliation address, remove the [a].
	
	% Use \vita if required to give biographical details (for authors of
	% invited review papers only). Uncomment it.
	
	% lca IUCr id IUCr6401
	%\vita{Author's biography}
	
	% Keywords (required for Journal of Synchrotron Radiation only)
	% Use the \keyword macro for each word or phrase, e.g. 
	% \keyword{X-ray diffraction}\keyword{muscle}
	
	
	% PDB and NDB reference codes for structures referenced in the article and
	% deposited with the Protein Data Bank and Nucleic Acids Database (Acta
	% Crystallographica Section D). Repeat for each separate structure e.g
	% \PDBref[dethiobiotin synthetase]{1byi} \NDBref[d(G$_4$CGC$_4$)]{ad0002}
	
	%\PDBref[optional name]{refcode}
	%\NDBref[optional name]{refcode}
	
	%-------------------------------------------------------------------------
	% The introductory (header) part of the paper
	%-------------------------------------------------------------------------
	
	% The title of the paper. Use \shorttitle to indicate an abbreviated title
	% for use in running heads (you will need to uncomment it).
	\begin{center}
		{\LARGE \emph{\today}} \\
	\end{center}
	
	\title{\PIII, reducing unit cell parameters}
	\shorttitle{reduction in \PIII}
	
	% Authors' names and addresses. Use \cauthor for the main (contact) author.
	% Use \author for all other authors. Use \aff for authors' affiliations.
	% Use lower-case letters in square brackets to link authors to their
	% affiliations; if there is only one affiliation address, remove the [a].
	
	
	\cauthor[a]{Lawrence C.}{Andrews}{larry6640995@gmail.com}{}
	\author[b]{Herbert J.}{Bernstein}
	
\aff[a]{Ronin Institute for Independent Scholarship 2.0, USA}
\aff[b]{Ronin Institute for Independent Scholarship 2.0, USA}
	
	% Use \shortauthor to indicate an abbreviated author list for use in
	% running heads (you will need to uncomment it).
	
	\shortauthor{Andrews and Bernstein}
	
	% Use \vita if required to give biographical details (for authors of
	% invited review papers only). Uncomment it.
	
	% lca IUCr id IUCr6401
	%\vita{Author's biography}
	
	% Keywords (required for Journal of Synchrotron Radiation only)
	% Use the \keyword macro for each word or phrase, e.g. 
	% \keyword{X-ray diffraction}\keyword{muscle}
	
	\keyword{lattice}
	\keyword{unit cell}
	\keyword{polar}
	\keyword{\PIII}
	
	
	\maketitle                        % DO NOT DELETE THIS LINE
	
	\begin{synopsis}
		
	\end{synopsis}
	
	
	\newcommand{\si}{\ensuremath{s_1}}
	\newcommand{\sii}{\ensuremath{s_2}}
	\newcommand{\siii}{\ensuremath{s_3}}
	\newcommand{\siv}{\ensuremath{s_4}}
	\newcommand{\sv}{\ensuremath{s_5}}
	\newcommand{\svi}{\ensuremath{s_6}}
	
	% Scalar vectors
	\newcommand{\Svec}{\ensuremath{\{ \si, \sii, \siii, \siv, \sv, \svi \}}}
	\newcommand{\SvecA}{\ensuremath{\{ -\si, -\si+\sii, \si+\siii, \si+\sv, \si+\siv, \si+\svi \}}}
	
	% Operators / maps
	\newcommand{\OPES}{\ensuremath{E^3\!\to\!S^6}}
	\newcommand{\OPESS}{
		
		\[ E^3\!\to\!S^6 \]
		
	}
	\newcommand{\MSVI}{\ensuremath{M_{S^6}}}
	\newcommand{\MEIII}{\ensuremath{M_{E^3}}}
	
	% Symbolic markers
	\newcommand{\Plus}{\ensuremath{\mathcal{P}}}
	\newcommand{\Minus}{\ensuremath{\mathcal{M}}}
	
	
	
	\begin{abstract}


	\end{abstract}
	% Appendices appear after the main body of the text. They are prefixed by
	% a single \appendix declaration, and are then structured just like the
	% body text.
	
	
	\section{Introduction}
	
	In crystallography, unit cell geometry is conventionally encoded in a six-dimensional parameter space 
	\ensuremath{\HVI{}=(a,b,c, \alpha, \beta, \gamma).}\footnote{``H'' was chosen to
		honor early French crystallographer Ren\'e Just Ha\"uy.} However, \HVI{} is not a metric space
	(where distances can be simply defined), nor is \HVI{} a vector space (where 
	objects can be added and subtracted). In the same sense, symmetry operations
	have no simple representations in \HVI{}. In consequence, \HVI{} has no
	simple measure of the ``difference'' between two lattices; how can one 
	compare a difference of 1 \AA{}ngstrom unit vs. one angular degree?
	


\section{	\PIII{} Is introduced.}


\section{Scalar Reduction and Canonical Presentation in Projected Vector Triplets}

\subsection{1. Standard Presentation of Projected Vectors}
	
	Let a set of three projected vectors be given:
	
	
	\[
	(p_1, p_2, p_3), \quad p_i \in \mathbb{R}^2
	\]
	
	
	Each \( p_i \) is constructed from a lattice vector via magnitude and opposing angle:
	
	
	\[
	p_i = |\vec{v}_i| (\cos \theta_i, \sin \theta_i)
	\]
	
	
	
	The \textbf{Standard Presentation} algorithm assigns a canonical ordering and orientation to the triplet:
	
	\begin{enumerate}
		\item \textbf{Magnitude Sort:} Reorder vectors so that
		
		
		\[
		|p_1| \leq |p_2| \leq |p_3|
		\]
		
		
		using Euclidean norms \( |p_i| = \sqrt{x_i^2 + y_i^2} \). Ties are preserved up to a numerical tolerance \( \delta \).
		
\item \textbf{Directional Coherence:} If the \( x \)-components of the triplet are not all of consistent sign, flip the necessary vectors by negating only the \( x_i \) component:


\[
p_i \to (-x_i,\ y_i)
\]


This maps quadrant I vectors into quadrant II, preserving their placement in the upper half-plane and maintaining angular conventions typical in crystallographic analysis.



\[
p_i \to (-x_i,\ y_i)
\]


to align the triplet into a directionally consistent state. If all \( x_i < 0 \), the configuration is already coherent and requires no change.

		
		
		\[
		p_i \to (-x_i, y_i)
		\]
		
		
		for any two vectors, preserving chirality.
		
		\item \textbf{Return Ordered Triplet:} Output the permuted and flipped triplet with consistent magnitude order and cosine directionality.
	\end{enumerate}
	
	\vspace{1em}
	
	\subsection{2. Scalar Projection Interaction and Cost Function}
	
	For any pair of projected vectors \( p_i, p_j \in \mathbb{R}^2 \), define the scalar projection interaction (or directional affinity):
	
	
	\[
	\mu_{ij} = p_i \cdot p_j = x_i x_j + y_i y_j
	\]
	
	
	
	The \textbf{scalar coupling cost} is then defined over the triplet as:
	
	
	\[
	C' = \frac{|\mu_{12}| + |\mu_{13}| + |\mu_{23}|}{p_1 \cdot p_1 + p_2 \cdot p_2 + p_3 \cdot p_3}
	\]
	
	
	
	This cost measures the normalized sum of pairwise directional alignment. Lower values of \( C' \) indicate that the projected vectors are less directionally coupled — i.e., more orthogonal or dispersed.
	
	\vspace{1em}
	
	\subsection{3. Scalar Reduction Algorithm}
	
	The goal of scalar reduction is to transform the triplet \( (p_1, p_2, p_3) \) into an algebraically simpler form — one with lower scalar coupling cost \( C' \) — while preserving lattice identity (e.g., volume and reconstructability).
	
	The reduction proceeds iteratively:
	
	\begin{enumerate}
		\item \textbf{Initialization:}
		Apply standard presentation. Compute initial cost \( C'_0 \) and physical volume \( V_0 > 0 \).
		
		\item \textbf{Pairwise Projection Subtractions:}
		For each unordered pair \( (p_i, p_j),\ i \neq j \), compute the scalar coefficient:
		
		
		\[
		\lambda = \frac{p_i \cdot p_j}{p_j \cdot p_j}
		\]
		
		
		Then define the candidate vector:
		
		
		\[
		p_i' = p_i - \lambda p_j
		\]
		
		
		
		\item \textbf{Evaluate Candidate:}
		Apply standard presentation to updated triplet. Recompute cost \( C'_{\text{new}} \) and volume \( V_{\text{new}} \).
		
		\item \textbf{Acceptance Criteria:}
		Accept the transformation if:
		
		
		\[
		C'_{\text{new}} + \delta < C' \quad \text{and} \quad V_{\text{new}} > 0
		\]
		
		
		
		\item \textbf{Repeat:}
		Continue until no further accepted transformations lower the cost.
		
		\item \textbf{Output:}
		Return the reduced triplet \( (p_1, p_2, p_3)_{\text{red}} \) with minimal scalar cost.
	\end{enumerate}
	
	\vspace{1em}
	
	\subsection{4. Interpretation of Cost and Reduction}
	
	The term "reduction" refers to minimizing the scalar coupling among vectors — reducing directional alignment that obscures presentation clarity. The cost \( C' \) is a rigorously defined metric for this alignment. Transformations reduce \( C' \) while preserving:
	\begin{itemize}
		\item The vector magnitudes and their reconstructable relationships
		\item The unit cell's volume and crystallographic identity
		\item The triplet’s algebraic validity in scalar projection space
	\end{itemize}
	
	Reduction leads to a cleaner, more canonical representation of the original structure — just as Niggli reduction provides a minimal cell in metric tensor space, scalar reduction provides a minimal configuration in projected vector space.
	
	
	
	
	\section{Summary}
	
	
	\section{Availability of code} CmdToP3 and PlotPolar are available in github.com, in
	\url{https://github.com/duck10/LatticeRepLib.git}.
	
	%\appendix
	
	
	%\section{blah blah blah -- Supplementary Material}
	\ack{{\bf Acknowledgements}}
	
	Careful copy-editing and corrections by Frances C. Bernstein are 
	gratefully acknowledged.
	%	Our thanks to Jean Jakoncic and Alexei Soares for 
	%	helpful conversations and access to data and facilities at 
	%	Brookhaven National Laboratory.
	%	
	\ack{{\bf Funding information}}      
	
	Funding for this research was provided in part by:  
	US Department of Energy Offices of Biological and 
	Environmental Research and of Basic Energy Sciences 
	(grant No. DE-AC02-98CH10886; grant No. E-SC0012704); 
	U.S. National Institutes of Health (grant No. P41RR012408; 
	grant No. P41GM103473; grant No. P41GM111244; 
	grant No. R01GM117126,
	grant No. 1R21GM129570); Dectris, Ltd.
	
	
	\bibliography{Reduced}
	
	\bibliographystyle{iucr}
	
	
	
	%-------------------------------------------------------------------------
	% TABLES AND FIGURES SHOULD BE INSERTED AFTER THE MAIN BODY OF THE TEXT
	%-------------------------------------------------------------------------
	
	% Simple tables should use the tabular environment according to this
	% model
	
	% Postscript figures can be included with multiple figure blocks
	
	%C:\Users\lca\Source\Repos\LatticeRepLib\x64\Debug>plotc3
	%; Graphical output SVG file =PLT__2023-03-07.13_43_35.svg
	%
	%C:\Users\lca\Source\Repos\LatticeRepLib\x64\Debug>cmdniggli | plotc3
	%; Graphical output SVG file =PLT__2023-03-07.13_44_06.svg
	%
	%C:\Users\lca\Source\Repos\LatticeRepLib\x64\Debug>cmddelone | plotc3
	%; Graphical output SVG file =PLT__2023-03-08.07_11_03.svg
	%
	%C:\Users\lca\Source\Repos\LatticeRepLib\x64\Debug>cmdniggli | cmdperturb 5 20 | plotc3
	%; Graphical output SVG file =PLT__2023-03-08.09_00_13.svg
	%
	%C:\Users\lca\Source\Repos\LatticeRepLib\x64\Debug>cmdniggli | cmdperturb 5 100 | plotc3
	%; Graphical output SVG file =PLT__2023-03-08.09_00_21.svg
	
\end{document}                    % DO NOT DELETE THIS LINE
%%%%%%%%%%%%%%%%%%%%%%%%%%%%%%%%%%%%%%%%%%%%%%%%%%%%%%%%%%%%%%%%%%%%%%%%%%%%%%
