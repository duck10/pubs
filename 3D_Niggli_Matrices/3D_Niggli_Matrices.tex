%------------------------------------------------------------------------------
% Template file for the submission of papers to IUCr journals in LaTeX2e
% using the iucr document class
% Copyright 1999-2013 International Union of Crystallography
% Version 1.6 (28 March 2013)
%------------------------------------------------------------------------------
% lca IUCr id IUCr6401
% HJB IUCr id IUCr6484
% NKS IUCr ID: IUCr7572
% lca ORCID  0000-0002-4451-1641
% HJB ORCID 0000-0002-0517-8532
% NKS ORCID 0000-0003-2786-6552
%xxxxxx just for testing github % 2
%xxxxxx just for testing github % 3

\documentclass[preprint]{iucr}              % DO NOT DELETE THIS LINE
\usepackage{amssymb}
\usepackage[fleqn]{amsmath}
%\usepackage{bm}
\usepackage{graphicx}
\usepackage{tabularx}
\usepackage{booktabs}
\usepackage{longtable}
\usepackage{makecell}
\usepackage{minibox}
%\usepackage{calligra}
\usepackage{array}
\DeclareMathAlphabet{\mathcalligra}{T1}{calligra}{m}{n}
\def\mathbi#1{\textbf{\em #1}}
\numberwithin{equation}{section}
%\DeclareMathSymbol{\Gamma}{\mathalpha}{letters}{"00}
%\DeclareMathSymbol{\Lambda}{\mathalpha}{letters}{"03}
%\DeclareMathSymbol{\Omega}{\mathalpha}{letters}{"0A}
%\DeclareMathAlphabet{\mathitbf}{OML}{cmm}{b}{it}
\hyphenation{Niggli}
\def\mathbi#1{\textbf{\em #1}}
%\numberwithin{equation}{section}
%\DeclareMathSymbol{\Gamma}{\mathalpha}{letters}{"00}
%\DeclareMathSymbol{\Lambda}{\mathalpha}{letters}{"03}
%\DeclareMathSymbol{\Omega}{\mathalpha}{letters}{"0A}
%\DeclareMathAlphabet{\mathitbf}{OML}{cmm}{b}{it}
\usepackage{color}
%\usepackage{ulem}
\usepackage{url}
\usepackage{yfonts}
%\usepackage{xr-hyper}
%\usepackage[draft]{hyperref}
%\usepackage{bibentry}
\usepackage[normalem]{ulem}
% Redefine \underline to use \textit
\renewcommand{\underline}[1]{\textit{#1}}
\newcommand{\SVI}[0]{$\bf{S^{6}}$}
\newcommand{\GVI}[0]{$\bf{G^{6}}$}
\newcommand{\CIII}[0]{$\bf{C^{3}}$}
\newcommand{\DVII}[0]{$\bf{D^{7}}$}
\newcommand{\VVII}[0]{$\bf{V^{7}}$}

\newcommand{\vdotv}[2]{${{\bf #1 \cdot #2}}$}
\newcommand{\Imaginary}[0]{\mathcal{I}}
\newcommand{\Real}[0]{\mathcal{R}}
\newcommand{\Exchange}[0]{$\mathcal{X}$}

\newcommand{\nounderline}[3]{\!\!\!\!\!\!\!\!\!#1,&\!\!\!\!\!\!\!\!\!#2,&\!\!\!\!\!\!\!\!\!#3}
\newcommand{\underlineab}[3]{\!\!\!\!\!\!\!\!\!\!\!\!\!\!\!\!\!\!\!\!\!\!\!\!\Exchange{}(#1),&\!\!\!\!\!\!\!\!\!\!\!\!\!\!\!\!\!\!\!\!\!\!\!\!\Exchange{}(#2),&\!\!\!\!\!\!\!\!\!#3}
\newcommand{\underlineac}[3]{\!\!\!\!\!\!\!\!\!\!\!\!\!\!\!\!\!\!\!\!\!\!\!\!\Exchange{}(#1),&\!\!\!\!\!\!\!\!\!\!\!\!\!\!\!\!\!\!\!\!\!\!\!\!#2,&\!\!\!\!\!\!\!\!\!\Exchange{}(#3)}
\newcommand{\underlinebc}[3]{\!\!\!\!\!\!\!\!\!\!\!\!\!\!\!\!\!\!\!\!\!\!\!\!#1,&\!\!\!\!\!\!\!\!\!\Exchange{}(#2),&\!\!\!\!\!\!\!\!\!\!\!\!\!\!\!\!\!\!\!\!\!\!\!\!\Exchange{}(#3)}

\newcommand{\scalar}[1]{$#1$}

\newcommand{\scalarsub}[2]{$#1_#2$}

%-------------------------------------------------------------------------
% Information about journal to which submitted
%-------------------------------------------------------------------------
\journalcode{A}              % Indicate the journal to which submitted
%   A - Acta Crystallographica Section A
%   B - Acta Crystallographica Section B
%   C - Acta Crystallographica Section C
%   D - Acta Crystallographica Section D
%   E - Acta Crystallographica Section E
%   F - Acta Crystallographica Section F
%   J - Journal of Applied Crystallography
%   M - IUCrJ
%   S - Journal of Synchrotron Radiation
\makeatletter
\font\dummyft@=dummy \relax
\makeatother


\begin{document}
		{\LARGE \emph{\today}} \\
	
\title{Niggli reduction: matrices for 3-D lattice transformations}

% Authors' names and addresses. Use \cauthor for the main (contact) author.
% Use \author for all other authors. Use \aff for authors' affiliations.
% Use lower-case letters in square brackets to link authors to their
% affiliations; if there is only one affiliation address, remove the [a].


\cauthor[a]{Lawrence C.}{Andrews}{larry6640995@gmail.com}{}
\author[b]{Herbert J.}{Bernstein}

\aff[a]{Ronin Institute for Independent Scholarship 2.0, USA}
\aff[b]{Ronin Institute for Independent Scholarship 2.0, USA}

% Use \shortauthor to indicate an abbreviated author list for use in
% running heads (you will need to uncomment it).

\shortauthor{Andrews and Bernstein}
	% Keywords (required for Journal of Synchrotron Radiation only)
% Use the \keyword macro for each word or phrase, e.g. 
% \keyword{X-ray diffraction}\keyword{muscle}

\keyword{lattice}
\keyword{reduction}
\keyword{Niggli}

	\maketitle
	\begin{abstract}
		Niggli reduction finds the most compact unit cell of a 3-D lattice.  
		Existing algorithms operate in \GVI{} space, and the 3-D basis-vector changes are lost.  
		We tabulate the 3×3 matrices for every step of the reduction, enabling direct re-indexing of reflections and full reconstruction of the reduction history.
		
	\end{abstract}
	
		\begin{synopsis}

	\end{synopsis}
	
	\section{Introduction}
	Niggli reduction~\cite{Niggli1928} finds the cell with the three shortest base vectors in a 3-D lattice. \citeasnoun{Gruber1973} published a compact algorithm using scalars that were
	later described in the space \GVI{} \cite{Andrews2014}. Although effective, 
	the algorithm works in a six-dimensional space, losing the three-space transformations
	needed for changing unit cell presentation or for reindexing reflection data.
	
	  Here we tabulate the 3-D matrices for transformation of unit cell basis vectors corresponding to each step
	  in Niggli reduction. They allow the accumulation during Niggli
	  reduction of all of the changes in the unit cell presentation.
	  The matrix that is finally accumulated transforms the input basis
	  to the Niggli reduced basis.
	
	\section{Conventions}
		\citeasnoun{Andrews2014} give the formal presentation of \GVI{}.
	\begin{itemize}
		\item $G^{6}=(g_{1},g_{2},g_{3},g_{4},g_{5},g_{6})$ with
		\[
		g_{1}=a^{2},\; g_{2}=b^{2},\; g_{3}=c^{2},\; g_{4}=2bc\cos\alpha,\; g_{5}=2ac\cos\beta,\; g_{6}=2ab\cos\gamma.
		\]
		\item \citeasnoun{Gruber1973}, called the process of putting the scalars
		into a fixed order ``normalizing''. We use the more
		descriptive term ``standard presentation''.
		\item Gruber labeled his pseudocode for standard presentation
		as ``Algorithm N'' and his pseudocode for reduction as 
		``Algorithm B''.

	\end{itemize}
	
\section{Matrices and Trigger Conditions}

We list matrices to transform unit cell basis vectors during 
Niggli reduction in Tables \ref{standardpresentation} and \ref{reduction},
along with arbitrary names for the operations and also the conditions
for triggering each operation during reduction. If a condition
evaluates to true, then the operation should be performed. For
complete Niggli reduction, the steps are repeated from the identity, SP0,
to convergence. 



The numbering of all
the operations, both standard presentation and reduction, start from zero
for the first standard presentation and continues through to R12. (Note that
the matrices for operations R9, R10 and R11 are the same as those for R5,
R6, and R7, respectively.)

\section{Usage}
\begin{enumerate}
	\item Start with SP0.
	\item Loop through SP and R tables until no trigger fires.
	\item Accumulate the product of all applied matrices; the final product maps the initial basis to the Niggli-reduced basis. 
\end{enumerate}



\begin{table}
	\caption{The matrices for Standard Presentation (SP). The 
	conditions for SP34a,b,c are patterns of $(g_4, g_5, g_6)$. The symbols are ``0'' for a fixed value of zero, ``+'' for a positive value in that position, and ``-'' for negative.}
	\label{standardpresentation}
	\centering
	\renewcommand{\arraystretch}{1.4}
	\begin{tabular}{@{}c@{\quad}l@{\quad}l@{}}
		\toprule
		\textbf{Designation} & \textbf{Matrix} & \textbf{Trigger condition on }$G^{6}$\\
		\midrule
		SP0 & $\begin{bmatrix}1&0&0\\0&1&0\\0&0&1\end{bmatrix}$ & (identity)\\
	\midrule
		SP1 & $\begin{bmatrix}0&-1&0\\-1&0&0\\0&0&-1\end{bmatrix}$ & $|g_{1}|>|g_{2}|+\delta$\\
	\midrule
		SP2 & $\begin{bmatrix}-1&0&0\\0&0&-1\\0&-1&0\end{bmatrix}$ & $|g_{2}|>|g_{3}|+\delta$\\
	\midrule
		SP34a & $\begin{bmatrix}-1&0&0\\0&-1&0\\0&0&1\end{bmatrix}$ & $(--+)~or~(+0-)~or~(0{+}-)$\\
		SP34b & $\begin{bmatrix}-1&0&0\\0&1&0\\0&0&-1\end{bmatrix}$ & $(-{+}-) ~or~ (+-0)~or~(0-+)$\\		SP34c & $\begin{bmatrix}1&0&0\\0&-1&0\\0&0&-1\end{bmatrix}$ & $(+--)~or~ (-{+}0)~or~(-0+)$\\
		\bottomrule
	\end{tabular}
\end{table}


\begin{longtable}[l]{@{}l@{\quad}l@{\quad}l@{}}
	\caption{The operations for Niggli reduction}
	\label{reduction}\\
	\toprule
	\textbf{Designation} & \textbf{Matrix} & \textbf{Trigger condition on }$G^{6}$ ($\delta>0$)\\
	\midrule
	\endfirsthead
	
	\multicolumn{3}{@{}c@{}}{\textit{— continued from previous page —}}\\
	\toprule
	\textbf{Designation} & \textbf{Matrix} & \textbf{Trigger condition on }$G^{6}$ ($\delta>0$)\\
	\midrule
	\endhead
	
	\midrule
	\multicolumn{3}{@{}c@{}}{\textit{— continued on next page —}}\\
	\endfoot
	\bottomrule
	\endlastfoot
	
	R5$^{+}$ & $\begin{bmatrix}1&0&0\\0&1&0\\0&-1&1\end{bmatrix}$
	&  $g_{4}>g_{2}+\delta$, $g_{4} >$ 0.0\\[2ex]
	
	R5$^{-}$ & $\begin{bmatrix}1&0&0\\0&1&0\\0&1&1\end{bmatrix}$
	&  $g_{4}>g_{2}+\delta$, $g_{4} \leq$ 0.0\\[2ex]
	\midrule
		
	R6$^{+}$ & $\begin{bmatrix}1&0&0\\0&1&0\\-1&0&1\end{bmatrix}$
	&  $g_{5}>g_{1}+\delta$, $g_{5} > 0$\\[2ex]
	
	R6$^{-}$ & $\begin{bmatrix}1&0&0\\0&1&0\\1&0&1\end{bmatrix}$
	&  $g_{5}>g_{1}+\delta$, $g_{5} <= 0$\\[2ex]
	\midrule
	R7$^{+}$ & $\begin{bmatrix}1&0&0\\-1&1&0\\0&0&1\end{bmatrix}$
	&  $g_{6}>g_{1}+\delta$, $g_{6} > 0$\\[2ex]
	
	R7$^{-}$ & $\begin{bmatrix}1&0&0\\1&1&0\\0&0&1\end{bmatrix}$
	&  $g_{6}>g_{1}+\delta$, $g_{6} <= 0$\\[2ex]
	\midrule
	R8 & $\begin{bmatrix}1&0&0\\0&1&0\\1&1&1\end{bmatrix}$
	&  $g_{4}+g_{5}+g_{6}+g{1}+g_{2}< -\delta$\\[2ex]\\
	\midrule
	
	R9$^{+}$ & same as R5$^{+}$
	&  $|g_{4}-g_{2}| <= \delta~ and~ 2g_{5} < g_{6}+\delta ~and~ g_{4} > 0$\\[2ex]
	
	R9$^{-}$ & same as R5$^{-}$
	&  $|g_{4}-g_{2}| <= \delta~ and~ 2g_{5} < g_{6}+\delta ~and~ g_{4} <= 0$\\[2ex]
	\midrule
	
	R10$^{+}$ & same as R6$^{+}$
	&  $|g_{5}-g_{1}| <= \delta~ and~ 2g_{4} < g_{6}+\delta ~and~ g_{5} > 0$\\[2ex]
	
	R10$^{-}$ & same as R6$^{-}$
	&  $|g_{5}-g_{1}| <= \delta~ and~ 2g_{4} < g_{6}+\delta ~and~ g_{5} <= 0$\\[2ex]

	\midrule

		R11$^{+}$ & same as R7$^{+}$ &
\minibox[l]{%
	$|g_{6}-g_{1}| \le \delta \text{ and } 2g_{4} < g_{5}+\delta \text{ and } g_{6} > 0$ \\
	$\text{or}$ \\
	$|g_{6}-g_{1}| < \delta \text{ and } 2g_{4} < g_{5} + \delta
	\text{ and } g_{6} > 0$%
} \\[4ex]\

		R11$^{-}$ & same as R7$^{-}$ &
\minibox[l]{%
	$|g_{6}-g_{1}| \le \delta \text{ and } 2g_{4} < g_{5}+\delta \text{ and } g_{6} <= 0$ \\
	$\text{or}$ \\
	$|g_{6}-g_{1}| < \delta \text{ and } 2g_{4} < g_{5} + \delta
	\text{ and } g_{6} <= 0$%
} \\[4ex]

	\midrule

	R12 & $\begin{bmatrix}1&0&0\\0&1&0\\-1&-1&1\end{bmatrix}$
&  $|g_{4}+g_{5}+g_{6}+g_{1} +g_{2}| <= \delta~ and~ 2|g_{1}+g_{5}|+g_{6} > \delta$\\[2ex]


%      else if (fabs(vin[3] + vin[4] + vin[5] + fabs(vin[0]) + fabs(vin[1])) <= delta &&
%(2.0 * (fabs(vin[0]) + vin[4]) + vin[5] > delta)) { // R12
%	m1 = R12;  // Use the global R12
%	m3d_step = R12_3D;
%	again = true;
	
	
\end{longtable}

	
\section{Synopsis}

Niggli reduction is formulated in terms of 6 scalars that formally are the components of space \GVI{}. Working in a
higher dimensional space loses the 3-space transformations that
are needed for reindexing reflection data and for changing
the orientation of atomic structure. The matrices are listed
here corresponding to each of the operations of Niggli reduction.

	\ack{{\bf Acknowledgements}}

We gratefully acknowledge the careful copy-editing and corrections of Frances C. Bernstein.

\ack{{\bf Funding information}}      

Funding for this research was provided in part by:  
US Department of Energy Offices of Biological and
Environmental Research and of Basic Energy Sciences
(grant No. DE-AC02-98CH10886; grant No. E-SC0012704);
U.S. National Institutes of Health (grant No. P41RR012408;
grant No. P41GM103473; grant No. P41GM111244;
grant No. R01GM117126,
grant No. 1R21GM129570); Dectris, Ltd, and in part by
Grant No. 1R24GM154040-01 from the NIH
and the Regents' Prime contract no. DE-AC02-05CH11231 from the Department of Energy.


\bibliography{Reduced}

\bibliographystyle{iucr}
	
%	\bibliographystyle{plain}f
%	\begin{thebibliography}{9}
%		\bibitem{niggli} Niggli, P. (1927). \emph{Handbuch der Experimentalphysik}, Leipzig.
%		\bibitem{gruber} Gruber, B. (1973). \emph{Acta Cryst.} A29, 433--440.
%		\bibitem{ab2014} Andrews, L. C. \& Bernstein, H. J. (2014). \emph{Acta Cryst.} A70, 411--418.
%	\end{thebibliography}
	
\end{document}